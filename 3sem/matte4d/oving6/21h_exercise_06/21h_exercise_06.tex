\documentclass[11pt]{article}

    \usepackage[breakable]{tcolorbox}
    \usepackage{parskip} % Stop auto-indenting (to mimic markdown behaviour)
    
    \usepackage{iftex}
    \ifPDFTeX
    	\usepackage[T1]{fontenc}
    	\usepackage{mathpazo}
    \else
    	\usepackage{fontspec}
    \fi

    % Basic figure setup, for now with no caption control since it's done
    % automatically by Pandoc (which extracts ![](path) syntax from Markdown).
    \usepackage{graphicx}
    % Maintain compatibility with old templates. Remove in nbconvert 6.0
    \let\Oldincludegraphics\includegraphics
    % Ensure that by default, figures have no caption (until we provide a
    % proper Figure object with a Caption API and a way to capture that
    % in the conversion process - todo).
    \usepackage{caption}
    \DeclareCaptionFormat{nocaption}{}
    \captionsetup{format=nocaption,aboveskip=0pt,belowskip=0pt}

    \usepackage{float}
    \floatplacement{figure}{H} % forces figures to be placed at the correct location
    \usepackage{xcolor} % Allow colors to be defined
    \usepackage{enumerate} % Needed for markdown enumerations to work
    \usepackage{geometry} % Used to adjust the document margins
    \usepackage{amsmath} % Equations
    \usepackage{amssymb} % Equations
    \usepackage{textcomp} % defines textquotesingle
    % Hack from http://tex.stackexchange.com/a/47451/13684:
    \AtBeginDocument{%
        \def\PYZsq{\textquotesingle}% Upright quotes in Pygmentized code
    }
    \usepackage{upquote} % Upright quotes for verbatim code
    \usepackage{eurosym} % defines \euro
    \usepackage[mathletters]{ucs} % Extended unicode (utf-8) support
    \usepackage{fancyvrb} % verbatim replacement that allows latex
    \usepackage{grffile} % extends the file name processing of package graphics 
                         % to support a larger range
    \makeatletter % fix for old versions of grffile with XeLaTeX
    \@ifpackagelater{grffile}{2019/11/01}
    {
      % Do nothing on new versions
    }
    {
      \def\Gread@@xetex#1{%
        \IfFileExists{"\Gin@base".bb}%
        {\Gread@eps{\Gin@base.bb}}%
        {\Gread@@xetex@aux#1}%
      }
    }
    \makeatother
    \usepackage[Export]{adjustbox} % Used to constrain images to a maximum size
    \adjustboxset{max size={0.9\linewidth}{0.9\paperheight}}

    % The hyperref package gives us a pdf with properly built
    % internal navigation ('pdf bookmarks' for the table of contents,
    % internal cross-reference links, web links for URLs, etc.)
    \usepackage{hyperref}
    % The default LaTeX title has an obnoxious amount of whitespace. By default,
    % titling removes some of it. It also provides customization options.
    \usepackage{titling}
    \usepackage{longtable} % longtable support required by pandoc >1.10
    \usepackage{booktabs}  % table support for pandoc > 1.12.2
    \usepackage[inline]{enumitem} % IRkernel/repr support (it uses the enumerate* environment)
    \usepackage[normalem]{ulem} % ulem is needed to support strikethroughs (\sout)
                                % normalem makes italics be italics, not underlines
    \usepackage{mathrsfs}
    

    
    % Colors for the hyperref package
    \definecolor{urlcolor}{rgb}{0,.145,.698}
    \definecolor{linkcolor}{rgb}{.71,0.21,0.01}
    \definecolor{citecolor}{rgb}{.12,.54,.11}

    % ANSI colors
    \definecolor{ansi-black}{HTML}{3E424D}
    \definecolor{ansi-black-intense}{HTML}{282C36}
    \definecolor{ansi-red}{HTML}{E75C58}
    \definecolor{ansi-red-intense}{HTML}{B22B31}
    \definecolor{ansi-green}{HTML}{00A250}
    \definecolor{ansi-green-intense}{HTML}{007427}
    \definecolor{ansi-yellow}{HTML}{DDB62B}
    \definecolor{ansi-yellow-intense}{HTML}{B27D12}
    \definecolor{ansi-blue}{HTML}{208FFB}
    \definecolor{ansi-blue-intense}{HTML}{0065CA}
    \definecolor{ansi-magenta}{HTML}{D160C4}
    \definecolor{ansi-magenta-intense}{HTML}{A03196}
    \definecolor{ansi-cyan}{HTML}{60C6C8}
    \definecolor{ansi-cyan-intense}{HTML}{258F8F}
    \definecolor{ansi-white}{HTML}{C5C1B4}
    \definecolor{ansi-white-intense}{HTML}{A1A6B2}
    \definecolor{ansi-default-inverse-fg}{HTML}{FFFFFF}
    \definecolor{ansi-default-inverse-bg}{HTML}{000000}

    % common color for the border for error outputs.
    \definecolor{outerrorbackground}{HTML}{FFDFDF}

    % commands and environments needed by pandoc snippets
    % extracted from the output of `pandoc -s`
    \providecommand{\tightlist}{%
      \setlength{\itemsep}{0pt}\setlength{\parskip}{0pt}}
    \DefineVerbatimEnvironment{Highlighting}{Verbatim}{commandchars=\\\{\}}
    % Add ',fontsize=\small' for more characters per line
    \newenvironment{Shaded}{}{}
    \newcommand{\KeywordTok}[1]{\textcolor[rgb]{0.00,0.44,0.13}{\textbf{{#1}}}}
    \newcommand{\DataTypeTok}[1]{\textcolor[rgb]{0.56,0.13,0.00}{{#1}}}
    \newcommand{\DecValTok}[1]{\textcolor[rgb]{0.25,0.63,0.44}{{#1}}}
    \newcommand{\BaseNTok}[1]{\textcolor[rgb]{0.25,0.63,0.44}{{#1}}}
    \newcommand{\FloatTok}[1]{\textcolor[rgb]{0.25,0.63,0.44}{{#1}}}
    \newcommand{\CharTok}[1]{\textcolor[rgb]{0.25,0.44,0.63}{{#1}}}
    \newcommand{\StringTok}[1]{\textcolor[rgb]{0.25,0.44,0.63}{{#1}}}
    \newcommand{\CommentTok}[1]{\textcolor[rgb]{0.38,0.63,0.69}{\textit{{#1}}}}
    \newcommand{\OtherTok}[1]{\textcolor[rgb]{0.00,0.44,0.13}{{#1}}}
    \newcommand{\AlertTok}[1]{\textcolor[rgb]{1.00,0.00,0.00}{\textbf{{#1}}}}
    \newcommand{\FunctionTok}[1]{\textcolor[rgb]{0.02,0.16,0.49}{{#1}}}
    \newcommand{\RegionMarkerTok}[1]{{#1}}
    \newcommand{\ErrorTok}[1]{\textcolor[rgb]{1.00,0.00,0.00}{\textbf{{#1}}}}
    \newcommand{\NormalTok}[1]{{#1}}
    
    % Additional commands for more recent versions of Pandoc
    \newcommand{\ConstantTok}[1]{\textcolor[rgb]{0.53,0.00,0.00}{{#1}}}
    \newcommand{\SpecialCharTok}[1]{\textcolor[rgb]{0.25,0.44,0.63}{{#1}}}
    \newcommand{\VerbatimStringTok}[1]{\textcolor[rgb]{0.25,0.44,0.63}{{#1}}}
    \newcommand{\SpecialStringTok}[1]{\textcolor[rgb]{0.73,0.40,0.53}{{#1}}}
    \newcommand{\ImportTok}[1]{{#1}}
    \newcommand{\DocumentationTok}[1]{\textcolor[rgb]{0.73,0.13,0.13}{\textit{{#1}}}}
    \newcommand{\AnnotationTok}[1]{\textcolor[rgb]{0.38,0.63,0.69}{\textbf{\textit{{#1}}}}}
    \newcommand{\CommentVarTok}[1]{\textcolor[rgb]{0.38,0.63,0.69}{\textbf{\textit{{#1}}}}}
    \newcommand{\VariableTok}[1]{\textcolor[rgb]{0.10,0.09,0.49}{{#1}}}
    \newcommand{\ControlFlowTok}[1]{\textcolor[rgb]{0.00,0.44,0.13}{\textbf{{#1}}}}
    \newcommand{\OperatorTok}[1]{\textcolor[rgb]{0.40,0.40,0.40}{{#1}}}
    \newcommand{\BuiltInTok}[1]{{#1}}
    \newcommand{\ExtensionTok}[1]{{#1}}
    \newcommand{\PreprocessorTok}[1]{\textcolor[rgb]{0.74,0.48,0.00}{{#1}}}
    \newcommand{\AttributeTok}[1]{\textcolor[rgb]{0.49,0.56,0.16}{{#1}}}
    \newcommand{\InformationTok}[1]{\textcolor[rgb]{0.38,0.63,0.69}{\textbf{\textit{{#1}}}}}
    \newcommand{\WarningTok}[1]{\textcolor[rgb]{0.38,0.63,0.69}{\textbf{\textit{{#1}}}}}
    
    
    % Define a nice break command that doesn't care if a line doesn't already
    % exist.
    \def\br{\hspace*{\fill} \\* }
    % Math Jax compatibility definitions
    \def\gt{>}
    \def\lt{<}
    \let\Oldtex\TeX
    \let\Oldlatex\LaTeX
    \renewcommand{\TeX}{\textrm{\Oldtex}}
    \renewcommand{\LaTeX}{\textrm{\Oldlatex}}
    % Document parameters
    % Document title
    \title{21h\_exercise\_06}
    
    
    
    
    
% Pygments definitions
\makeatletter
\def\PY@reset{\let\PY@it=\relax \let\PY@bf=\relax%
    \let\PY@ul=\relax \let\PY@tc=\relax%
    \let\PY@bc=\relax \let\PY@ff=\relax}
\def\PY@tok#1{\csname PY@tok@#1\endcsname}
\def\PY@toks#1+{\ifx\relax#1\empty\else%
    \PY@tok{#1}\expandafter\PY@toks\fi}
\def\PY@do#1{\PY@bc{\PY@tc{\PY@ul{%
    \PY@it{\PY@bf{\PY@ff{#1}}}}}}}
\def\PY#1#2{\PY@reset\PY@toks#1+\relax+\PY@do{#2}}

\@namedef{PY@tok@w}{\def\PY@tc##1{\textcolor[rgb]{0.73,0.73,0.73}{##1}}}
\@namedef{PY@tok@c}{\let\PY@it=\textit\def\PY@tc##1{\textcolor[rgb]{0.25,0.50,0.50}{##1}}}
\@namedef{PY@tok@cp}{\def\PY@tc##1{\textcolor[rgb]{0.74,0.48,0.00}{##1}}}
\@namedef{PY@tok@k}{\let\PY@bf=\textbf\def\PY@tc##1{\textcolor[rgb]{0.00,0.50,0.00}{##1}}}
\@namedef{PY@tok@kp}{\def\PY@tc##1{\textcolor[rgb]{0.00,0.50,0.00}{##1}}}
\@namedef{PY@tok@kt}{\def\PY@tc##1{\textcolor[rgb]{0.69,0.00,0.25}{##1}}}
\@namedef{PY@tok@o}{\def\PY@tc##1{\textcolor[rgb]{0.40,0.40,0.40}{##1}}}
\@namedef{PY@tok@ow}{\let\PY@bf=\textbf\def\PY@tc##1{\textcolor[rgb]{0.67,0.13,1.00}{##1}}}
\@namedef{PY@tok@nb}{\def\PY@tc##1{\textcolor[rgb]{0.00,0.50,0.00}{##1}}}
\@namedef{PY@tok@nf}{\def\PY@tc##1{\textcolor[rgb]{0.00,0.00,1.00}{##1}}}
\@namedef{PY@tok@nc}{\let\PY@bf=\textbf\def\PY@tc##1{\textcolor[rgb]{0.00,0.00,1.00}{##1}}}
\@namedef{PY@tok@nn}{\let\PY@bf=\textbf\def\PY@tc##1{\textcolor[rgb]{0.00,0.00,1.00}{##1}}}
\@namedef{PY@tok@ne}{\let\PY@bf=\textbf\def\PY@tc##1{\textcolor[rgb]{0.82,0.25,0.23}{##1}}}
\@namedef{PY@tok@nv}{\def\PY@tc##1{\textcolor[rgb]{0.10,0.09,0.49}{##1}}}
\@namedef{PY@tok@no}{\def\PY@tc##1{\textcolor[rgb]{0.53,0.00,0.00}{##1}}}
\@namedef{PY@tok@nl}{\def\PY@tc##1{\textcolor[rgb]{0.63,0.63,0.00}{##1}}}
\@namedef{PY@tok@ni}{\let\PY@bf=\textbf\def\PY@tc##1{\textcolor[rgb]{0.60,0.60,0.60}{##1}}}
\@namedef{PY@tok@na}{\def\PY@tc##1{\textcolor[rgb]{0.49,0.56,0.16}{##1}}}
\@namedef{PY@tok@nt}{\let\PY@bf=\textbf\def\PY@tc##1{\textcolor[rgb]{0.00,0.50,0.00}{##1}}}
\@namedef{PY@tok@nd}{\def\PY@tc##1{\textcolor[rgb]{0.67,0.13,1.00}{##1}}}
\@namedef{PY@tok@s}{\def\PY@tc##1{\textcolor[rgb]{0.73,0.13,0.13}{##1}}}
\@namedef{PY@tok@sd}{\let\PY@it=\textit\def\PY@tc##1{\textcolor[rgb]{0.73,0.13,0.13}{##1}}}
\@namedef{PY@tok@si}{\let\PY@bf=\textbf\def\PY@tc##1{\textcolor[rgb]{0.73,0.40,0.53}{##1}}}
\@namedef{PY@tok@se}{\let\PY@bf=\textbf\def\PY@tc##1{\textcolor[rgb]{0.73,0.40,0.13}{##1}}}
\@namedef{PY@tok@sr}{\def\PY@tc##1{\textcolor[rgb]{0.73,0.40,0.53}{##1}}}
\@namedef{PY@tok@ss}{\def\PY@tc##1{\textcolor[rgb]{0.10,0.09,0.49}{##1}}}
\@namedef{PY@tok@sx}{\def\PY@tc##1{\textcolor[rgb]{0.00,0.50,0.00}{##1}}}
\@namedef{PY@tok@m}{\def\PY@tc##1{\textcolor[rgb]{0.40,0.40,0.40}{##1}}}
\@namedef{PY@tok@gh}{\let\PY@bf=\textbf\def\PY@tc##1{\textcolor[rgb]{0.00,0.00,0.50}{##1}}}
\@namedef{PY@tok@gu}{\let\PY@bf=\textbf\def\PY@tc##1{\textcolor[rgb]{0.50,0.00,0.50}{##1}}}
\@namedef{PY@tok@gd}{\def\PY@tc##1{\textcolor[rgb]{0.63,0.00,0.00}{##1}}}
\@namedef{PY@tok@gi}{\def\PY@tc##1{\textcolor[rgb]{0.00,0.63,0.00}{##1}}}
\@namedef{PY@tok@gr}{\def\PY@tc##1{\textcolor[rgb]{1.00,0.00,0.00}{##1}}}
\@namedef{PY@tok@ge}{\let\PY@it=\textit}
\@namedef{PY@tok@gs}{\let\PY@bf=\textbf}
\@namedef{PY@tok@gp}{\let\PY@bf=\textbf\def\PY@tc##1{\textcolor[rgb]{0.00,0.00,0.50}{##1}}}
\@namedef{PY@tok@go}{\def\PY@tc##1{\textcolor[rgb]{0.53,0.53,0.53}{##1}}}
\@namedef{PY@tok@gt}{\def\PY@tc##1{\textcolor[rgb]{0.00,0.27,0.87}{##1}}}
\@namedef{PY@tok@err}{\def\PY@bc##1{{\setlength{\fboxsep}{\string -\fboxrule}\fcolorbox[rgb]{1.00,0.00,0.00}{1,1,1}{\strut ##1}}}}
\@namedef{PY@tok@kc}{\let\PY@bf=\textbf\def\PY@tc##1{\textcolor[rgb]{0.00,0.50,0.00}{##1}}}
\@namedef{PY@tok@kd}{\let\PY@bf=\textbf\def\PY@tc##1{\textcolor[rgb]{0.00,0.50,0.00}{##1}}}
\@namedef{PY@tok@kn}{\let\PY@bf=\textbf\def\PY@tc##1{\textcolor[rgb]{0.00,0.50,0.00}{##1}}}
\@namedef{PY@tok@kr}{\let\PY@bf=\textbf\def\PY@tc##1{\textcolor[rgb]{0.00,0.50,0.00}{##1}}}
\@namedef{PY@tok@bp}{\def\PY@tc##1{\textcolor[rgb]{0.00,0.50,0.00}{##1}}}
\@namedef{PY@tok@fm}{\def\PY@tc##1{\textcolor[rgb]{0.00,0.00,1.00}{##1}}}
\@namedef{PY@tok@vc}{\def\PY@tc##1{\textcolor[rgb]{0.10,0.09,0.49}{##1}}}
\@namedef{PY@tok@vg}{\def\PY@tc##1{\textcolor[rgb]{0.10,0.09,0.49}{##1}}}
\@namedef{PY@tok@vi}{\def\PY@tc##1{\textcolor[rgb]{0.10,0.09,0.49}{##1}}}
\@namedef{PY@tok@vm}{\def\PY@tc##1{\textcolor[rgb]{0.10,0.09,0.49}{##1}}}
\@namedef{PY@tok@sa}{\def\PY@tc##1{\textcolor[rgb]{0.73,0.13,0.13}{##1}}}
\@namedef{PY@tok@sb}{\def\PY@tc##1{\textcolor[rgb]{0.73,0.13,0.13}{##1}}}
\@namedef{PY@tok@sc}{\def\PY@tc##1{\textcolor[rgb]{0.73,0.13,0.13}{##1}}}
\@namedef{PY@tok@dl}{\def\PY@tc##1{\textcolor[rgb]{0.73,0.13,0.13}{##1}}}
\@namedef{PY@tok@s2}{\def\PY@tc##1{\textcolor[rgb]{0.73,0.13,0.13}{##1}}}
\@namedef{PY@tok@sh}{\def\PY@tc##1{\textcolor[rgb]{0.73,0.13,0.13}{##1}}}
\@namedef{PY@tok@s1}{\def\PY@tc##1{\textcolor[rgb]{0.73,0.13,0.13}{##1}}}
\@namedef{PY@tok@mb}{\def\PY@tc##1{\textcolor[rgb]{0.40,0.40,0.40}{##1}}}
\@namedef{PY@tok@mf}{\def\PY@tc##1{\textcolor[rgb]{0.40,0.40,0.40}{##1}}}
\@namedef{PY@tok@mh}{\def\PY@tc##1{\textcolor[rgb]{0.40,0.40,0.40}{##1}}}
\@namedef{PY@tok@mi}{\def\PY@tc##1{\textcolor[rgb]{0.40,0.40,0.40}{##1}}}
\@namedef{PY@tok@il}{\def\PY@tc##1{\textcolor[rgb]{0.40,0.40,0.40}{##1}}}
\@namedef{PY@tok@mo}{\def\PY@tc##1{\textcolor[rgb]{0.40,0.40,0.40}{##1}}}
\@namedef{PY@tok@ch}{\let\PY@it=\textit\def\PY@tc##1{\textcolor[rgb]{0.25,0.50,0.50}{##1}}}
\@namedef{PY@tok@cm}{\let\PY@it=\textit\def\PY@tc##1{\textcolor[rgb]{0.25,0.50,0.50}{##1}}}
\@namedef{PY@tok@cpf}{\let\PY@it=\textit\def\PY@tc##1{\textcolor[rgb]{0.25,0.50,0.50}{##1}}}
\@namedef{PY@tok@c1}{\let\PY@it=\textit\def\PY@tc##1{\textcolor[rgb]{0.25,0.50,0.50}{##1}}}
\@namedef{PY@tok@cs}{\let\PY@it=\textit\def\PY@tc##1{\textcolor[rgb]{0.25,0.50,0.50}{##1}}}

\def\PYZbs{\char`\\}
\def\PYZus{\char`\_}
\def\PYZob{\char`\{}
\def\PYZcb{\char`\}}
\def\PYZca{\char`\^}
\def\PYZam{\char`\&}
\def\PYZlt{\char`\<}
\def\PYZgt{\char`\>}
\def\PYZsh{\char`\#}
\def\PYZpc{\char`\%}
\def\PYZdl{\char`\$}
\def\PYZhy{\char`\-}
\def\PYZsq{\char`\'}
\def\PYZdq{\char`\"}
\def\PYZti{\char`\~}
% for compatibility with earlier versions
\def\PYZat{@}
\def\PYZlb{[}
\def\PYZrb{]}
\makeatother


    % For linebreaks inside Verbatim environment from package fancyvrb. 
    \makeatletter
        \newbox\Wrappedcontinuationbox 
        \newbox\Wrappedvisiblespacebox 
        \newcommand*\Wrappedvisiblespace {\textcolor{red}{\textvisiblespace}} 
        \newcommand*\Wrappedcontinuationsymbol {\textcolor{red}{\llap{\tiny$\m@th\hookrightarrow$}}} 
        \newcommand*\Wrappedcontinuationindent {3ex } 
        \newcommand*\Wrappedafterbreak {\kern\Wrappedcontinuationindent\copy\Wrappedcontinuationbox} 
        % Take advantage of the already applied Pygments mark-up to insert 
        % potential linebreaks for TeX processing. 
        %        {, <, #, %, $, ' and ": go to next line. 
        %        _, }, ^, &, >, - and ~: stay at end of broken line. 
        % Use of \textquotesingle for straight quote. 
        \newcommand*\Wrappedbreaksatspecials {% 
            \def\PYGZus{\discretionary{\char`\_}{\Wrappedafterbreak}{\char`\_}}% 
            \def\PYGZob{\discretionary{}{\Wrappedafterbreak\char`\{}{\char`\{}}% 
            \def\PYGZcb{\discretionary{\char`\}}{\Wrappedafterbreak}{\char`\}}}% 
            \def\PYGZca{\discretionary{\char`\^}{\Wrappedafterbreak}{\char`\^}}% 
            \def\PYGZam{\discretionary{\char`\&}{\Wrappedafterbreak}{\char`\&}}% 
            \def\PYGZlt{\discretionary{}{\Wrappedafterbreak\char`\<}{\char`\<}}% 
            \def\PYGZgt{\discretionary{\char`\>}{\Wrappedafterbreak}{\char`\>}}% 
            \def\PYGZsh{\discretionary{}{\Wrappedafterbreak\char`\#}{\char`\#}}% 
            \def\PYGZpc{\discretionary{}{\Wrappedafterbreak\char`\%}{\char`\%}}% 
            \def\PYGZdl{\discretionary{}{\Wrappedafterbreak\char`\$}{\char`\$}}% 
            \def\PYGZhy{\discretionary{\char`\-}{\Wrappedafterbreak}{\char`\-}}% 
            \def\PYGZsq{\discretionary{}{\Wrappedafterbreak\textquotesingle}{\textquotesingle}}% 
            \def\PYGZdq{\discretionary{}{\Wrappedafterbreak\char`\"}{\char`\"}}% 
            \def\PYGZti{\discretionary{\char`\~}{\Wrappedafterbreak}{\char`\~}}% 
        } 
        % Some characters . , ; ? ! / are not pygmentized. 
        % This macro makes them "active" and they will insert potential linebreaks 
        \newcommand*\Wrappedbreaksatpunct {% 
            \lccode`\~`\.\lowercase{\def~}{\discretionary{\hbox{\char`\.}}{\Wrappedafterbreak}{\hbox{\char`\.}}}% 
            \lccode`\~`\,\lowercase{\def~}{\discretionary{\hbox{\char`\,}}{\Wrappedafterbreak}{\hbox{\char`\,}}}% 
            \lccode`\~`\;\lowercase{\def~}{\discretionary{\hbox{\char`\;}}{\Wrappedafterbreak}{\hbox{\char`\;}}}% 
            \lccode`\~`\:\lowercase{\def~}{\discretionary{\hbox{\char`\:}}{\Wrappedafterbreak}{\hbox{\char`\:}}}% 
            \lccode`\~`\?\lowercase{\def~}{\discretionary{\hbox{\char`\?}}{\Wrappedafterbreak}{\hbox{\char`\?}}}% 
            \lccode`\~`\!\lowercase{\def~}{\discretionary{\hbox{\char`\!}}{\Wrappedafterbreak}{\hbox{\char`\!}}}% 
            \lccode`\~`\/\lowercase{\def~}{\discretionary{\hbox{\char`\/}}{\Wrappedafterbreak}{\hbox{\char`\/}}}% 
            \catcode`\.\active
            \catcode`\,\active 
            \catcode`\;\active
            \catcode`\:\active
            \catcode`\?\active
            \catcode`\!\active
            \catcode`\/\active 
            \lccode`\~`\~ 	
        }
    \makeatother

    \let\OriginalVerbatim=\Verbatim
    \makeatletter
    \renewcommand{\Verbatim}[1][1]{%
        %\parskip\z@skip
        \sbox\Wrappedcontinuationbox {\Wrappedcontinuationsymbol}%
        \sbox\Wrappedvisiblespacebox {\FV@SetupFont\Wrappedvisiblespace}%
        \def\FancyVerbFormatLine ##1{\hsize\linewidth
            \vtop{\raggedright\hyphenpenalty\z@\exhyphenpenalty\z@
                \doublehyphendemerits\z@\finalhyphendemerits\z@
                \strut ##1\strut}%
        }%
        % If the linebreak is at a space, the latter will be displayed as visible
        % space at end of first line, and a continuation symbol starts next line.
        % Stretch/shrink are however usually zero for typewriter font.
        \def\FV@Space {%
            \nobreak\hskip\z@ plus\fontdimen3\font minus\fontdimen4\font
            \discretionary{\copy\Wrappedvisiblespacebox}{\Wrappedafterbreak}
            {\kern\fontdimen2\font}%
        }%
        
        % Allow breaks at special characters using \PYG... macros.
        \Wrappedbreaksatspecials
        % Breaks at punctuation characters . , ; ? ! and / need catcode=\active 	
        \OriginalVerbatim[#1,codes*=\Wrappedbreaksatpunct]%
    }
    \makeatother

    % Exact colors from NB
    \definecolor{incolor}{HTML}{303F9F}
    \definecolor{outcolor}{HTML}{D84315}
    \definecolor{cellborder}{HTML}{CFCFCF}
    \definecolor{cellbackground}{HTML}{F7F7F7}
    
    % prompt
    \makeatletter
    \newcommand{\boxspacing}{\kern\kvtcb@left@rule\kern\kvtcb@boxsep}
    \makeatother
    \newcommand{\prompt}[4]{
        {\ttfamily\llap{{\color{#2}[#3]:\hspace{3pt}#4}}\vspace{-\baselineskip}}
    }
    

    
    % Prevent overflowing lines due to hard-to-break entities
    \sloppy 
    % Setup hyperref package
    \hypersetup{
      breaklinks=true,  % so long urls are correctly broken across lines
      colorlinks=true,
      urlcolor=urlcolor,
      linkcolor=linkcolor,
      citecolor=citecolor,
      }
    % Slightly bigger margins than the latex defaults
    
    \geometry{verbose,tmargin=1in,bmargin=1in,lmargin=1in,rmargin=1in}
    
    

\begin{document}
    
    \maketitle
    
    

    
    \hypertarget{exercises-6-numerical-solution-of-odes-1}{%
\section{Exercises 6: Numerical Solution of ODE's
1}\label{exercises-6-numerical-solution-of-odes-1}}

TMA4130 - Mathematics 4N / TMA4135 - Mathematics 4D

\begin{itemize}
\tightlist
\item
  Date: \textbf{Oct 6, 2021}
\item
  Submission deadline: \textbf{Oct 20, 2021}
\end{itemize}

If you want to have a nicer theme for your jupyter notebook, download
the
\href{https://www.math.ntnu.no/emner/TMA4130/2021h/lectures/calculus4N.css}{cascade
stylesheet file calculus4N.css} and execute the next cell:

    \begin{tcolorbox}[breakable, size=fbox, boxrule=1pt, pad at break*=1mm,colback=cellbackground, colframe=cellborder]
\prompt{In}{incolor}{1}{\boxspacing}
\begin{Verbatim}[commandchars=\\\{\}]
\PY{k+kn}{from} \PY{n+nn}{IPython}\PY{n+nn}{.}\PY{n+nn}{core}\PY{n+nn}{.}\PY{n+nn}{display} \PY{k+kn}{import} \PY{n}{HTML}
\PY{k}{def} \PY{n+nf}{css\PYZus{}styling}\PY{p}{(}\PY{p}{)}\PY{p}{:}
    \PY{k}{try}\PY{p}{:}
        \PY{k}{with} \PY{n+nb}{open}\PY{p}{(}\PY{l+s+s2}{\PYZdq{}}\PY{l+s+s2}{calculus4N.css}\PY{l+s+s2}{\PYZdq{}}\PY{p}{,} \PY{l+s+s2}{\PYZdq{}}\PY{l+s+s2}{r}\PY{l+s+s2}{\PYZdq{}}\PY{p}{)} \PY{k}{as} \PY{n}{f}\PY{p}{:}
            \PY{n}{styles} \PY{o}{=} \PY{n}{f}\PY{o}{.}\PY{n}{read}\PY{p}{(}\PY{p}{)}
            \PY{k}{return} \PY{n}{HTML}\PY{p}{(}\PY{n}{styles}\PY{p}{)}
    \PY{k}{except} \PY{n+ne}{FileNotFoundError}\PY{p}{:}
        \PY{n+nb}{print}\PY{p}{(}\PY{l+s+s2}{\PYZdq{}}\PY{l+s+s2}{Not found}\PY{l+s+s2}{\PYZdq{}}\PY{p}{)}
        \PY{k}{pass} \PY{c+c1}{\PYZsh{}Do nothing}

\PY{c+c1}{\PYZsh{} Comment out next line and execute this cell to restore the default notebook style }
\PY{n}{css\PYZus{}styling}\PY{p}{(}\PY{p}{)}
\end{Verbatim}
\end{tcolorbox}

            \begin{tcolorbox}[breakable, size=fbox, boxrule=.5pt, pad at break*=1mm, opacityfill=0]
\prompt{Out}{outcolor}{1}{\boxspacing}
\begin{Verbatim}[commandchars=\\\{\}]
<IPython.core.display.HTML object>
\end{Verbatim}
\end{tcolorbox}
        
    \begin{tcolorbox}[breakable, size=fbox, boxrule=1pt, pad at break*=1mm,colback=cellbackground, colframe=cellborder]
\prompt{In}{incolor}{2}{\boxspacing}
\begin{Verbatim}[commandchars=\\\{\}]
\PY{k+kn}{import} \PY{n+nn}{matplotlib}
\PY{n}{matplotlib}\PY{o}{.}\PY{n}{rcParams}\PY{o}{.}\PY{n}{update}\PY{p}{(}\PY{p}{\PYZob{}}\PY{l+s+s1}{\PYZsq{}}\PY{l+s+s1}{font.size}\PY{l+s+s1}{\PYZsq{}}\PY{p}{:} \PY{l+m+mi}{12}\PY{p}{\PYZcb{}}\PY{p}{)}
\PY{k+kn}{import} \PY{n+nn}{matplotlib}\PY{n+nn}{.}\PY{n+nn}{pyplot} \PY{k}{as} \PY{n+nn}{plt}
\PY{k+kn}{import} \PY{n+nn}{numpy} \PY{k}{as} \PY{n+nn}{np}

\PY{n}{plt}\PY{o}{.}\PY{n}{xkcd}\PY{p}{(}\PY{p}{)}
\PY{n}{newparams} \PY{o}{=} \PY{p}{\PYZob{}}\PY{l+s+s1}{\PYZsq{}}\PY{l+s+s1}{figure.figsize}\PY{l+s+s1}{\PYZsq{}}\PY{p}{:} \PY{p}{(}\PY{l+m+mf}{6.0}\PY{p}{,} \PY{l+m+mf}{6.0}\PY{p}{)}\PY{p}{,} \PY{l+s+s1}{\PYZsq{}}\PY{l+s+s1}{axes.grid}\PY{l+s+s1}{\PYZsq{}}\PY{p}{:} \PY{k+kc}{True}\PY{p}{,}
             \PY{l+s+s1}{\PYZsq{}}\PY{l+s+s1}{lines.markersize}\PY{l+s+s1}{\PYZsq{}}\PY{p}{:} \PY{l+m+mi}{8}\PY{p}{,} \PY{l+s+s1}{\PYZsq{}}\PY{l+s+s1}{lines.linewidth}\PY{l+s+s1}{\PYZsq{}}\PY{p}{:} \PY{l+m+mi}{2}\PY{p}{,}
             \PY{l+s+s1}{\PYZsq{}}\PY{l+s+s1}{font.size}\PY{l+s+s1}{\PYZsq{}}\PY{p}{:} \PY{l+m+mi}{14}\PY{p}{\PYZcb{}}
\PY{n}{plt}\PY{o}{.}\PY{n}{rcParams}\PY{o}{.}\PY{n}{update}\PY{p}{(}\PY{n}{newparams}\PY{p}{)}
\end{Verbatim}
\end{tcolorbox}

    In this exercise set you will be analyzing and implementing the
following explicit Runge-Kutta methods:

Midpoint rule \[
\begin{array}{c|cc}
0 & 0 & 0 \\ 
\tfrac{1}{2} & \frac{1}{2} & 0 \\ \hline
& 0 & 1
\end{array}
\] Gottlieb \& Gottlieb's 3-stage Runge-Kutta (SSPRK3) \[
\begin{array}{c|cccc}
0 & 0 & 0 & 0 \\ 
1 & 1 & 0 & 0 \\ 
\tfrac{1}{2} & \tfrac{1}{4} & \tfrac{1}{4} & 0 \\
\hline
& \tfrac{1}{6} & \tfrac{1}{6} & \tfrac{2}{3} 
\end{array}
\]

    \hypertarget{exercise-1}{%
\subsubsection{\texorpdfstring{\textbf{Exercise
1}}{Exercise 1}}\label{exercise-1}}

\hypertarget{convergence-orders}{%
\paragraph{Convergence orders}\label{convergence-orders}}

Calculate analytically the convergence order's of the two methods. Use
the order's conditions given in the lectures.

    \begin{tcolorbox}[breakable, size=fbox, boxrule=1pt, pad at break*=1mm,colback=cellbackground, colframe=cellborder]
\prompt{In}{incolor}{3}{\boxspacing}
\begin{Verbatim}[commandchars=\\\{\}]
\PY{c+c1}{\PYZsh{} creates a function that will return the convergene order given a vector b, n x n matrice and a vector c}
\PY{k}{def} \PY{n+nf}{calculate\PYZus{}convergence}\PY{p}{(}\PY{n}{a}\PY{p}{,} \PY{n}{b}\PY{p}{,} \PY{n}{c}\PY{p}{)}\PY{p}{:}
    \PY{n}{n} \PY{o}{=} \PY{n+nb}{len}\PY{p}{(}\PY{n}{b}\PY{p}{)}
    \PY{k}{if} \PY{p}{(}\PY{n+nb}{sum}\PY{p}{(}\PY{n}{b}\PY{p}{)} \PY{o}{==} \PY{l+m+mi}{1}\PY{p}{)}\PY{p}{:}
        \PY{n}{sum\PYZus{}order2} \PY{o}{=} \PY{l+m+mi}{0}
        \PY{k}{for} \PY{n}{i} \PY{o+ow}{in} \PY{n+nb}{range}\PY{p}{(}\PY{n}{n}\PY{p}{)}\PY{p}{:}
            \PY{n}{sum\PYZus{}order2} \PY{o}{+}\PY{o}{=} \PY{n}{b}\PY{p}{[}\PY{n}{i}\PY{p}{]} \PY{o}{*} \PY{n}{c}\PY{p}{[}\PY{n}{i}\PY{p}{]}
            
        \PY{k}{if} \PY{p}{(}\PY{n}{sum\PYZus{}order2} \PY{o}{==} \PY{l+m+mi}{1}\PY{o}{/}\PY{l+m+mi}{2}\PY{p}{)}\PY{p}{:}
            \PY{n}{sum\PYZus{}order3\PYZus{}1} \PY{o}{=} \PY{l+m+mi}{0}
            \PY{n}{sum\PYZus{}order3\PYZus{}2} \PY{o}{=} \PY{l+m+mi}{0}
            \PY{k}{for} \PY{n}{i} \PY{o+ow}{in} \PY{n+nb}{range}\PY{p}{(}\PY{n}{n}\PY{p}{)}\PY{p}{:}
                \PY{n}{sum\PYZus{}order3\PYZus{}1} \PY{o}{+}\PY{o}{=} \PY{n}{b}\PY{p}{[}\PY{n}{i}\PY{p}{]} \PY{o}{*} \PY{n}{c}\PY{p}{[}\PY{n}{i}\PY{p}{]}\PY{o}{*}\PY{o}{*}\PY{l+m+mi}{2}
                \PY{k}{for} \PY{n}{j} \PY{o+ow}{in} \PY{n+nb}{range}\PY{p}{(}\PY{n}{n}\PY{p}{)}\PY{p}{:}
                    \PY{n}{sum\PYZus{}order3\PYZus{}2} \PY{o}{+}\PY{o}{=} \PY{n}{b}\PY{p}{[}\PY{n}{i}\PY{p}{]} \PY{o}{*} \PY{n}{a}\PY{p}{[}\PY{n}{i}\PY{p}{]}\PY{p}{[}\PY{n}{j}\PY{p}{]} \PY{o}{*} \PY{n}{c}\PY{p}{[}\PY{n}{j}\PY{p}{]}
            
            \PY{k}{if} \PY{p}{(}\PY{n}{sum\PYZus{}order3\PYZus{}1} \PY{o}{==} \PY{l+m+mi}{1}\PY{o}{/}\PY{l+m+mi}{3} \PY{o+ow}{and} \PY{n}{sum\PYZus{}order3\PYZus{}2} \PY{o}{==} \PY{l+m+mi}{1}\PY{o}{/}\PY{l+m+mi}{6}\PY{p}{)}\PY{p}{:}
                \PY{n}{sum\PYZus{}order4\PYZus{}1} \PY{o}{=} \PY{l+m+mi}{0}
                \PY{n}{sum\PYZus{}order4\PYZus{}2} \PY{o}{=} \PY{l+m+mi}{0}
                \PY{n}{sum\PYZus{}order4\PYZus{}3} \PY{o}{=} \PY{l+m+mi}{0}
                \PY{n}{sum\PYZus{}order4\PYZus{}4} \PY{o}{=} \PY{l+m+mi}{0}
                
                \PY{k}{for} \PY{n}{i} \PY{o+ow}{in} \PY{n+nb}{range}\PY{p}{(}\PY{n}{n}\PY{p}{)}\PY{p}{:}
                    \PY{n}{sum\PYZus{}order4\PYZus{}1} \PY{o}{+}\PY{o}{=} \PY{n}{b}\PY{p}{[}\PY{n}{i}\PY{p}{]} \PY{o}{*} \PY{n}{c}\PY{p}{[}\PY{n}{i}\PY{p}{]}\PY{o}{*}\PY{o}{*}\PY{l+m+mi}{3}
                    \PY{k}{for} \PY{n}{j} \PY{o+ow}{in} \PY{n+nb}{range}\PY{p}{(}\PY{n}{n}\PY{p}{)}\PY{p}{:}
                        \PY{n}{sum\PYZus{}order4\PYZus{}2} \PY{o}{+}\PY{o}{=} \PY{n}{b}\PY{p}{[}\PY{n}{i}\PY{p}{]} \PY{o}{*} \PY{n}{c}\PY{p}{[}\PY{n}{i}\PY{p}{]} \PY{o}{*} \PY{n}{a}\PY{p}{[}\PY{n}{i}\PY{p}{]}\PY{p}{[}\PY{n}{j}\PY{p}{]} \PY{o}{*} \PY{n}{c}\PY{p}{[}\PY{n}{j}\PY{p}{]}
                        \PY{n}{sum\PYZus{}order4\PYZus{}3} \PY{o}{+}\PY{o}{=} \PY{n}{b}\PY{p}{[}\PY{n}{i}\PY{p}{]} \PY{o}{*} \PY{n}{a}\PY{p}{[}\PY{n}{i}\PY{p}{]}\PY{p}{[}\PY{n}{j}\PY{p}{]} \PY{o}{*} \PY{n}{c}\PY{p}{[}\PY{n}{j}\PY{p}{]}\PY{o}{*}\PY{o}{*}\PY{l+m+mi}{2}
                        \PY{k}{for} \PY{n}{k} \PY{o+ow}{in} \PY{n+nb}{range}\PY{p}{(}\PY{n}{n}\PY{p}{)}\PY{p}{:}
                            \PY{n}{sum\PYZus{}order4\PYZus{}4} \PY{o}{+}\PY{o}{=} \PY{n}{b}\PY{p}{[}\PY{n}{i}\PY{p}{]} \PY{o}{*} \PY{n}{a}\PY{p}{[}\PY{n}{i}\PY{p}{]}\PY{p}{[}\PY{n}{j}\PY{p}{]} \PY{o}{*} \PY{n}{a}\PY{p}{[}\PY{n}{j}\PY{p}{]}\PY{p}{[}\PY{n}{k}\PY{p}{]} \PY{o}{*} \PY{n}{c}\PY{p}{[}\PY{n}{k}\PY{p}{]}
                \PY{k}{if} \PY{p}{(}\PY{n}{sum\PYZus{}order4\PYZus{}1} \PY{o}{==} \PY{l+m+mi}{1}\PY{o}{/}\PY{l+m+mi}{4} \PY{o+ow}{and} \PY{n}{sum\PYZus{}order4\PYZus{}2} \PY{o}{==} \PY{l+m+mi}{1}\PY{o}{/}\PY{l+m+mi}{8} \PY{o+ow}{and} \PY{n}{sum\PYZus{}order4\PYZus{}3} \PY{o}{==} \PY{l+m+mi}{1}\PY{o}{/}\PY{l+m+mi}{12} \PY{o+ow}{and} \PY{n}{sum\PYZus{}order4\PYZus{}4} \PY{o}{==} \PY{l+m+mi}{1}\PY{o}{/}\PY{l+m+mi}{24}\PY{p}{)}\PY{p}{:}
                    \PY{k}{return} \PY{l+m+mi}{4}
                \PY{k}{else}\PY{p}{:}
                    \PY{k}{return} \PY{l+m+mi}{3}       
            \PY{k}{else}\PY{p}{:}
                \PY{k}{return} \PY{l+m+mi}{2}
        \PY{k}{else}\PY{p}{:}
            \PY{k}{return} \PY{l+m+mi}{1}
    \PY{k}{else}\PY{p}{:}
        \PY{k}{return} \PY{l+m+mi}{0}
\end{Verbatim}
\end{tcolorbox}

    \begin{tcolorbox}[breakable, size=fbox, boxrule=1pt, pad at break*=1mm,colback=cellbackground, colframe=cellborder]
\prompt{In}{incolor}{4}{\boxspacing}
\begin{Verbatim}[commandchars=\\\{\}]
\PY{c+c1}{\PYZsh{} midpoint rule}
\PY{n}{a\PYZus{}m} \PY{o}{=} \PY{n}{np}\PY{o}{.}\PY{n}{array}\PY{p}{(}\PY{p}{[}\PY{p}{[}\PY{l+m+mi}{0}\PY{p}{,} \PY{l+m+mi}{0}\PY{p}{]}\PY{p}{,} \PY{p}{[}\PY{l+m+mi}{1}\PY{o}{/}\PY{l+m+mi}{2}\PY{p}{,} \PY{l+m+mi}{0}\PY{p}{]}\PY{p}{]}\PY{p}{)}
\PY{n}{b\PYZus{}m} \PY{o}{=} \PY{n}{np}\PY{o}{.}\PY{n}{array}\PY{p}{(}\PY{p}{[}\PY{l+m+mi}{0}\PY{p}{,} \PY{l+m+mi}{1}\PY{p}{]}\PY{p}{)}
\PY{n}{c\PYZus{}m} \PY{o}{=} \PY{n}{np}\PY{o}{.}\PY{n}{array}\PY{p}{(}\PY{p}{[}\PY{l+m+mi}{0}\PY{p}{,} \PY{l+m+mi}{1}\PY{o}{/}\PY{l+m+mi}{2}\PY{p}{]}\PY{p}{)}

\PY{n+nb}{print}\PY{p}{(}\PY{l+s+s2}{\PYZdq{}}\PY{l+s+s2}{Order of convergence for midpoint rule:}\PY{l+s+s2}{\PYZdq{}}\PY{p}{,} \PY{n}{calculate\PYZus{}convergence}\PY{p}{(}\PY{n}{a\PYZus{}m}\PY{p}{,} \PY{n}{b\PYZus{}m}\PY{p}{,} \PY{n}{c\PYZus{}m}\PY{p}{)}\PY{p}{)}

\PY{c+c1}{\PYZsh{} SSPRK3}
\PY{n}{a\PYZus{}ssp} \PY{o}{=} \PY{n}{np}\PY{o}{.}\PY{n}{array}\PY{p}{(}\PY{p}{[}\PY{p}{[}\PY{l+m+mi}{0}\PY{p}{,} \PY{l+m+mi}{0}\PY{p}{,} \PY{l+m+mi}{0}\PY{p}{]}\PY{p}{,} \PY{p}{[}\PY{l+m+mi}{1}\PY{p}{,} \PY{l+m+mi}{0}\PY{p}{,} \PY{l+m+mi}{0}\PY{p}{]}\PY{p}{,} \PY{p}{[}\PY{l+m+mi}{1}\PY{o}{/}\PY{l+m+mi}{4}\PY{p}{,} \PY{l+m+mi}{1}\PY{o}{/}\PY{l+m+mi}{4}\PY{p}{,} \PY{l+m+mi}{0}\PY{p}{]}\PY{p}{]}\PY{p}{)}
\PY{n}{b\PYZus{}ssp} \PY{o}{=} \PY{n}{np}\PY{o}{.}\PY{n}{array}\PY{p}{(}\PY{p}{[}\PY{l+m+mi}{1}\PY{o}{/}\PY{l+m+mi}{6}\PY{p}{,} \PY{l+m+mi}{1}\PY{o}{/}\PY{l+m+mi}{6}\PY{p}{,} \PY{l+m+mi}{2}\PY{o}{/}\PY{l+m+mi}{3}\PY{p}{]}\PY{p}{)}
\PY{n}{c\PYZus{}ssp} \PY{o}{=} \PY{n}{np}\PY{o}{.}\PY{n}{array}\PY{p}{(}\PY{p}{[}\PY{l+m+mi}{0}\PY{p}{,} \PY{l+m+mi}{1}\PY{p}{,} \PY{l+m+mi}{1}\PY{o}{/}\PY{l+m+mi}{2}\PY{p}{]}\PY{p}{)}

\PY{n+nb}{print}\PY{p}{(}\PY{l+s+s2}{\PYZdq{}}\PY{l+s+s2}{Order of convergence for SSPRK3:}\PY{l+s+s2}{\PYZdq{}}\PY{p}{,} \PY{n}{calculate\PYZus{}convergence}\PY{p}{(}\PY{n}{a\PYZus{}ssp}\PY{p}{,} \PY{n}{b\PYZus{}ssp}\PY{p}{,} \PY{n}{c\PYZus{}ssp}\PY{p}{)}\PY{p}{)}
\end{Verbatim}
\end{tcolorbox}

    \begin{Verbatim}[commandchars=\\\{\}]
Order of convergence for midpoint rule: 2
Order of convergence for SSPRK3: 3
    \end{Verbatim}

    \hypertarget{exercise-2}{%
\subsubsection{\texorpdfstring{\textbf{Exercise
2}}{Exercise 2}}\label{exercise-2}}

\hypertarget{implementing-and-testing-the-methods}{%
\paragraph{Implementing and testing the
methods}\label{implementing-and-testing-the-methods}}

In this exercise we will numerically solve the ODE \[
y'(t) = f(y), \quad y(0) = y_0
\] in the interval \(t \in [0,T]\).

    \textbf{a)} Implement two \texttt{Python} functions
\texttt{explicit\_mid\_point\_rule} and \texttt{ssprk3} which implement
the Runge-Kutta methods from Exercise 1. Each solver function should
take as arguments: * The initial value \(y_0\) * The inital time \(t_0\)
* The final time \(T\) * The right-hand side \(f\) * The maximum number
of time-steps \(N_{max}\)

The function should return two arrays: * One array \texttt{ts}
containing all the time-points\\
\(0 = t_0,t_1,...,t_N = T\) * One array \texttt{ys} containing all the
function values\\
\(y_0,y_1,...,y_N\)

Test the methods on the ODE \[
y'(t) = -y(t), \quad y(0) = 1, \quad t \in [0,10].
\]

\emph{\textbf{Hint:}} Use the code for \texttt{explicit\_euler} in the
lecture notes or use the
\href{https://wiki.math.ntnu.no/tma4130/2021h/learning_material}{supporting
material} e.g.~\texttt{Heun} in IntroductionNuMeODE, and modify it to
each required method.

    \begin{tcolorbox}[breakable, size=fbox, boxrule=1pt, pad at break*=1mm,colback=cellbackground, colframe=cellborder]
\prompt{In}{incolor}{5}{\boxspacing}
\begin{Verbatim}[commandchars=\\\{\}]
\PY{k}{def} \PY{n+nf}{explicit\PYZus{}midpoint\PYZus{}rule}\PY{p}{(}\PY{n}{y0}\PY{p}{,} \PY{n}{t0}\PY{p}{,} \PY{n}{T}\PY{p}{,} \PY{n}{f}\PY{p}{,} \PY{n}{N\PYZus{}max}\PY{p}{)}\PY{p}{:}
    \PY{n}{ts} \PY{o}{=} \PY{p}{[}\PY{n}{t0}\PY{p}{]}
    \PY{n}{ys} \PY{o}{=} \PY{p}{[}\PY{n}{y0}\PY{p}{]}
    \PY{n}{tau} \PY{o}{=} \PY{p}{(}\PY{n}{T} \PY{o}{\PYZhy{}} \PY{n}{t0}\PY{p}{)}\PY{o}{/}\PY{n}{N\PYZus{}max}
    
    \PY{c+c1}{\PYZsh{} butcher table}
    \PY{n}{a} \PY{o}{=} \PY{n}{np}\PY{o}{.}\PY{n}{array}\PY{p}{(}\PY{p}{[}\PY{p}{[}\PY{l+m+mi}{0}\PY{p}{,} \PY{l+m+mi}{0}\PY{p}{]}\PY{p}{,} \PY{p}{[}\PY{l+m+mi}{1}\PY{o}{/}\PY{l+m+mi}{2}\PY{p}{,} \PY{l+m+mi}{0}\PY{p}{]}\PY{p}{]}\PY{p}{)}
    \PY{n}{b} \PY{o}{=} \PY{n}{np}\PY{o}{.}\PY{n}{array}\PY{p}{(}\PY{p}{[}\PY{l+m+mi}{0}\PY{p}{,} \PY{l+m+mi}{1}\PY{p}{]}\PY{p}{)}
    \PY{n}{c} \PY{o}{=} \PY{n}{np}\PY{o}{.}\PY{n}{array}\PY{p}{(}\PY{p}{[}\PY{l+m+mi}{0}\PY{p}{,} \PY{l+m+mi}{1}\PY{o}{/}\PY{l+m+mi}{2}\PY{p}{]}\PY{p}{)}
    
    \PY{n}{s} \PY{o}{=} \PY{n+nb}{len}\PY{p}{(}\PY{n}{b}\PY{p}{)}
    \PY{n}{ks} \PY{o}{=} \PY{p}{[}\PY{n}{np}\PY{o}{.}\PY{n}{zeros\PYZus{}like}\PY{p}{(}\PY{n}{y0}\PY{p}{,} \PY{n}{dtype}\PY{o}{=}\PY{n}{np}\PY{o}{.}\PY{n}{double}\PY{p}{)} \PY{k}{for} \PY{n}{s} \PY{o+ow}{in} \PY{n+nb}{range}\PY{p}{(}\PY{n}{s}\PY{p}{)}\PY{p}{]}
    
    \PY{k}{while} \PY{p}{(}\PY{n}{ts}\PY{p}{[}\PY{o}{\PYZhy{}}\PY{l+m+mi}{1}\PY{p}{]} \PY{o}{\PYZlt{}} \PY{n}{T}\PY{p}{)}\PY{p}{:}
        \PY{n}{t}\PY{p}{,} \PY{n}{y} \PY{o}{=} \PY{n}{ts}\PY{p}{[}\PY{o}{\PYZhy{}}\PY{l+m+mi}{1}\PY{p}{]}\PY{p}{,} \PY{n}{ys}\PY{p}{[}\PY{o}{\PYZhy{}}\PY{l+m+mi}{1}\PY{p}{]}
        \PY{c+c1}{\PYZsh{} computing k\PYZus{}j\PYZsq{}s }
        \PY{k}{for} \PY{n}{j} \PY{o+ow}{in} \PY{n+nb}{range}\PY{p}{(}\PY{n}{s}\PY{p}{)}\PY{p}{:}
            \PY{n}{dY} \PY{o}{=} \PY{l+m+mi}{0}
            \PY{k}{for} \PY{n}{l} \PY{o+ow}{in} \PY{n+nb}{range}\PY{p}{(}\PY{n}{j}\PY{p}{)}\PY{p}{:}
                \PY{n}{dY} \PY{o}{+}\PY{o}{=} \PY{n}{a}\PY{p}{[}\PY{n}{j}\PY{p}{]}\PY{p}{[}\PY{n}{l}\PY{p}{]} \PY{o}{*} \PY{n}{ks}\PY{p}{[}\PY{n}{l}\PY{p}{]}
            \PY{n}{ks}\PY{p}{[}\PY{n}{j}\PY{p}{]} \PY{o}{=} \PY{n}{f}\PY{p}{(}\PY{n}{t} \PY{o}{+} \PY{n}{c}\PY{p}{[}\PY{n}{j}\PY{p}{]} \PY{o}{*} \PY{n}{tau}\PY{p}{,} \PY{n}{y} \PY{o}{+} \PY{n}{tau} \PY{o}{*} \PY{n}{dY}\PY{p}{)}
        \PY{n}{dY} \PY{o}{=} \PY{l+m+mi}{0}
        \PY{k}{for} \PY{n}{j} \PY{o+ow}{in} \PY{n+nb}{range}\PY{p}{(}\PY{n}{s}\PY{p}{)}\PY{p}{:}
            \PY{n}{dY} \PY{o}{+}\PY{o}{=} \PY{n}{b}\PY{p}{[}\PY{n}{j}\PY{p}{]}\PY{o}{*}\PY{n}{ks}\PY{p}{[}\PY{n}{j}\PY{p}{]}
        \PY{n}{ts}\PY{o}{.}\PY{n}{append}\PY{p}{(}\PY{n}{t} \PY{o}{+} \PY{n}{tau}\PY{p}{)}
        \PY{n}{ys}\PY{o}{.}\PY{n}{append}\PY{p}{(}\PY{n}{y} \PY{o}{+} \PY{n}{tau}\PY{o}{*}\PY{n}{dY}\PY{p}{)}
    
    \PY{k}{return} \PY{n}{ts}\PY{p}{,} \PY{n}{ys}

\PY{k}{def} \PY{n+nf}{explicit\PYZus{}ssprk3}\PY{p}{(}\PY{n}{y0}\PY{p}{,} \PY{n}{t0}\PY{p}{,} \PY{n}{T}\PY{p}{,} \PY{n}{f}\PY{p}{,} \PY{n}{N\PYZus{}max}\PY{p}{)}\PY{p}{:}
    \PY{c+c1}{\PYZsh{} initial values}
    \PY{n}{ts} \PY{o}{=} \PY{p}{[}\PY{n}{t0}\PY{p}{]}
    \PY{n}{ys} \PY{o}{=} \PY{p}{[}\PY{n}{y0}\PY{p}{]}
    \PY{n}{tau} \PY{o}{=} \PY{p}{(}\PY{n}{T} \PY{o}{\PYZhy{}} \PY{n}{t0}\PY{p}{)}\PY{o}{/}\PY{n}{N\PYZus{}max}
    
    \PY{c+c1}{\PYZsh{} butcher table}
    \PY{n}{a} \PY{o}{=} \PY{n}{np}\PY{o}{.}\PY{n}{array}\PY{p}{(}\PY{p}{[}\PY{p}{[}\PY{l+m+mi}{0}\PY{p}{,} \PY{l+m+mi}{0}\PY{p}{,} \PY{l+m+mi}{0}\PY{p}{]}\PY{p}{,} \PY{p}{[}\PY{l+m+mi}{1}\PY{p}{,} \PY{l+m+mi}{0}\PY{p}{,} \PY{l+m+mi}{0}\PY{p}{]}\PY{p}{,} \PY{p}{[}\PY{l+m+mi}{1}\PY{o}{/}\PY{l+m+mi}{4}\PY{p}{,} \PY{l+m+mi}{1}\PY{o}{/}\PY{l+m+mi}{4}\PY{p}{,} \PY{l+m+mi}{0}\PY{p}{]}\PY{p}{]}\PY{p}{)}
    \PY{n}{b} \PY{o}{=} \PY{n}{np}\PY{o}{.}\PY{n}{array}\PY{p}{(}\PY{p}{[}\PY{l+m+mi}{1}\PY{o}{/}\PY{l+m+mi}{6}\PY{p}{,} \PY{l+m+mi}{1}\PY{o}{/}\PY{l+m+mi}{6}\PY{p}{,} \PY{l+m+mi}{2}\PY{o}{/}\PY{l+m+mi}{3}\PY{p}{]}\PY{p}{)}
    \PY{n}{c} \PY{o}{=} \PY{n}{np}\PY{o}{.}\PY{n}{array}\PY{p}{(}\PY{p}{[}\PY{l+m+mi}{0}\PY{p}{,} \PY{l+m+mi}{1}\PY{p}{,} \PY{l+m+mi}{1}\PY{o}{/}\PY{l+m+mi}{2}\PY{p}{]}\PY{p}{)}
    
    \PY{n}{s} \PY{o}{=} \PY{n+nb}{len}\PY{p}{(}\PY{n}{b}\PY{p}{)}
    \PY{n}{ks} \PY{o}{=} \PY{p}{[}\PY{n}{np}\PY{o}{.}\PY{n}{zeros\PYZus{}like}\PY{p}{(}\PY{n}{y0}\PY{p}{,} \PY{n}{dtype}\PY{o}{=}\PY{n}{np}\PY{o}{.}\PY{n}{double}\PY{p}{)} \PY{k}{for} \PY{n}{s} \PY{o+ow}{in} \PY{n+nb}{range}\PY{p}{(}\PY{n}{s}\PY{p}{)}\PY{p}{]}
    
    \PY{k}{while} \PY{p}{(}\PY{n}{ts}\PY{p}{[}\PY{o}{\PYZhy{}}\PY{l+m+mi}{1}\PY{p}{]} \PY{o}{\PYZlt{}} \PY{n}{T}\PY{p}{)}\PY{p}{:}
        \PY{n}{t}\PY{p}{,} \PY{n}{y} \PY{o}{=} \PY{n}{ts}\PY{p}{[}\PY{o}{\PYZhy{}}\PY{l+m+mi}{1}\PY{p}{]}\PY{p}{,} \PY{n}{ys}\PY{p}{[}\PY{o}{\PYZhy{}}\PY{l+m+mi}{1}\PY{p}{]}
        \PY{c+c1}{\PYZsh{} computing k\PYZus{}j\PYZsq{}s }
        \PY{k}{for} \PY{n}{j} \PY{o+ow}{in} \PY{n+nb}{range}\PY{p}{(}\PY{n}{s}\PY{p}{)}\PY{p}{:}
            \PY{n}{dY} \PY{o}{=} \PY{n}{np}\PY{o}{.}\PY{n}{zeros\PYZus{}like}\PY{p}{(}\PY{n}{y}\PY{p}{,} \PY{n}{dtype}\PY{o}{=}\PY{n}{np}\PY{o}{.}\PY{n}{double}\PY{p}{)}
            \PY{k}{for} \PY{n}{l} \PY{o+ow}{in} \PY{n+nb}{range}\PY{p}{(}\PY{n}{j}\PY{p}{)}\PY{p}{:}
                \PY{n}{dY} \PY{o}{+}\PY{o}{=} \PY{n}{a}\PY{p}{[}\PY{n}{j}\PY{p}{]}\PY{p}{[}\PY{n}{l}\PY{p}{]} \PY{o}{*} \PY{n}{ks}\PY{p}{[}\PY{n}{l}\PY{p}{]}
            \PY{n}{ks}\PY{p}{[}\PY{n}{j}\PY{p}{]} \PY{o}{=} \PY{n}{f}\PY{p}{(}\PY{n}{t} \PY{o}{+} \PY{n}{c}\PY{p}{[}\PY{n}{j}\PY{p}{]} \PY{o}{*} \PY{n}{tau}\PY{p}{,} \PY{n}{y} \PY{o}{+} \PY{n}{tau} \PY{o}{*} \PY{n}{dY}\PY{p}{)}
        \PY{n}{dY} \PY{o}{=} \PY{n}{np}\PY{o}{.}\PY{n}{zeros\PYZus{}like}\PY{p}{(}\PY{n}{y}\PY{p}{,} \PY{n}{dtype}\PY{o}{=}\PY{n}{np}\PY{o}{.}\PY{n}{double}\PY{p}{)}
        \PY{k}{for} \PY{n}{j} \PY{o+ow}{in} \PY{n+nb}{range}\PY{p}{(}\PY{n}{s}\PY{p}{)}\PY{p}{:}
            \PY{n}{dY} \PY{o}{+}\PY{o}{=} \PY{n}{b}\PY{p}{[}\PY{n}{j}\PY{p}{]}\PY{o}{*}\PY{n}{ks}\PY{p}{[}\PY{n}{j}\PY{p}{]}
        \PY{n}{ts}\PY{o}{.}\PY{n}{append}\PY{p}{(}\PY{n}{t} \PY{o}{+} \PY{n}{tau}\PY{p}{)}
        \PY{n}{ys}\PY{o}{.}\PY{n}{append}\PY{p}{(}\PY{n}{y} \PY{o}{+} \PY{n}{tau}\PY{o}{*}\PY{n}{dY}\PY{p}{)}
    \PY{k}{return} \PY{n}{np}\PY{o}{.}\PY{n}{array}\PY{p}{(}\PY{n}{ts}\PY{p}{)}\PY{p}{,} \PY{n}{np}\PY{o}{.}\PY{n}{array}\PY{p}{(}\PY{n}{ys}\PY{p}{)}
\end{Verbatim}
\end{tcolorbox}

    \begin{tcolorbox}[breakable, size=fbox, boxrule=1pt, pad at break*=1mm,colback=cellbackground, colframe=cellborder]
\prompt{In}{incolor}{6}{\boxspacing}
\begin{Verbatim}[commandchars=\\\{\}]
\PY{n}{y0} \PY{o}{=} \PY{l+m+mi}{1}
\PY{n}{t0} \PY{o}{=} \PY{l+m+mi}{0}
\PY{n}{T} \PY{o}{=} \PY{l+m+mi}{10}
\PY{n}{N\PYZus{}max} \PY{o}{=} \PY{l+m+mi}{20}


\PY{k}{def} \PY{n+nf}{f}\PY{p}{(}\PY{n}{t}\PY{p}{,} \PY{n}{y}\PY{p}{)}\PY{p}{:}
    \PY{k}{return} \PY{o}{\PYZhy{}}\PY{n}{y}

\PY{c+c1}{\PYZsh{} exact solution (exp(\PYZhy{}t))}
\PY{k}{def} \PY{n+nf}{y\PYZus{}ex}\PY{p}{(}\PY{n}{t}\PY{p}{)}\PY{p}{:}
    \PY{k}{return} \PY{n}{np}\PY{o}{.}\PY{n}{exp}\PY{p}{(}\PY{o}{\PYZhy{}}\PY{n}{t}\PY{p}{)}

\PY{n}{fig}\PY{p}{,} \PY{n}{axes} \PY{o}{=} \PY{n}{plt}\PY{o}{.}\PY{n}{subplots}\PY{p}{(}\PY{l+m+mi}{1}\PY{p}{,}\PY{l+m+mi}{2}\PY{p}{)}
\PY{n}{ts}\PY{p}{,} \PY{n}{ys\PYZus{}mdr} \PY{o}{=} \PY{n}{explicit\PYZus{}midpoint\PYZus{}rule}\PY{p}{(}\PY{n}{y0}\PY{p}{,} \PY{n}{t0}\PY{p}{,} \PY{n}{T}\PY{p}{,} \PY{n}{f}\PY{p}{,} \PY{n}{N\PYZus{}max}\PY{p}{)}
\PY{n}{ys\PYZus{}ex} \PY{o}{=} \PY{p}{[}\PY{p}{]}
\PY{k}{for} \PY{n}{t} \PY{o+ow}{in} \PY{n}{ts}\PY{p}{:}
    \PY{n}{ys\PYZus{}ex}\PY{o}{.}\PY{n}{append}\PY{p}{(}\PY{n}{y\PYZus{}ex}\PY{p}{(}\PY{n}{t}\PY{p}{)}\PY{p}{)}
\PY{n}{axes}\PY{p}{[}\PY{l+m+mi}{0}\PY{p}{]}\PY{o}{.}\PY{n}{plot}\PY{p}{(}\PY{n}{ts}\PY{p}{,} \PY{n}{ys\PYZus{}ex}\PY{p}{,} \PY{l+s+s1}{\PYZsq{}}\PY{l+s+s1}{bo\PYZhy{}}\PY{l+s+s1}{\PYZsq{}}\PY{p}{)}
\PY{n}{axes}\PY{p}{[}\PY{l+m+mi}{0}\PY{p}{]}\PY{o}{.}\PY{n}{plot}\PY{p}{(}\PY{n}{ts}\PY{p}{,} \PY{n}{ys\PYZus{}mdr}\PY{p}{,} \PY{l+s+s1}{\PYZsq{}}\PY{l+s+s1}{rx\PYZhy{}}\PY{l+s+s1}{\PYZsq{}}\PY{p}{)}
\PY{n}{axes}\PY{p}{[}\PY{l+m+mi}{0}\PY{p}{]}\PY{o}{.}\PY{n}{legend}\PY{p}{(}\PY{p}{[}\PY{l+s+s2}{\PYZdq{}}\PY{l+s+s2}{\PYZdl{}y\PYZus{}}\PY{l+s+s2}{\PYZob{}}\PY{l+s+s2}{\PYZbs{}}\PY{l+s+s2}{mathrm}\PY{l+s+si}{\PYZob{}ex\PYZcb{}}\PY{l+s+s2}{\PYZcb{}\PYZdl{}}\PY{l+s+s2}{\PYZdq{}}\PY{p}{,} \PY{l+s+s2}{\PYZdq{}}\PY{l+s+s2}{\PYZdl{}y\PYZus{}}\PY{l+s+s2}{\PYZob{}}\PY{l+s+s2}{\PYZbs{}}\PY{l+s+s2}{mathrm}\PY{l+s+si}{\PYZob{}mpr\PYZcb{}}\PY{l+s+s2}{\PYZcb{}\PYZdl{}}\PY{l+s+s2}{\PYZdq{}} \PY{p}{]}\PY{p}{)}
\PY{n}{ts}\PY{p}{,} \PY{n}{ys\PYZus{}ssprk3} \PY{o}{=} \PY{n}{explicit\PYZus{}ssprk3}\PY{p}{(}\PY{n}{y0}\PY{p}{,} \PY{n}{t0}\PY{p}{,} \PY{n}{T}\PY{p}{,} \PY{n}{f}\PY{p}{,} \PY{n}{N\PYZus{}max}\PY{p}{)}
\PY{n}{axes}\PY{p}{[}\PY{l+m+mi}{1}\PY{p}{]}\PY{o}{.}\PY{n}{plot}\PY{p}{(}\PY{n}{ts}\PY{p}{,} \PY{n}{ys\PYZus{}ex}\PY{p}{,} \PY{l+s+s1}{\PYZsq{}}\PY{l+s+s1}{bo\PYZhy{}}\PY{l+s+s1}{\PYZsq{}}\PY{p}{)}
\PY{n}{axes}\PY{p}{[}\PY{l+m+mi}{1}\PY{p}{]}\PY{o}{.}\PY{n}{plot}\PY{p}{(}\PY{n}{ts}\PY{p}{,} \PY{n}{ys\PYZus{}ssprk3}\PY{p}{,} \PY{l+s+s1}{\PYZsq{}}\PY{l+s+s1}{rx\PYZhy{}}\PY{l+s+s1}{\PYZsq{}}\PY{p}{)}
\PY{n}{axes}\PY{p}{[}\PY{l+m+mi}{1}\PY{p}{]}\PY{o}{.}\PY{n}{legend}\PY{p}{(}\PY{p}{[}\PY{l+s+s2}{\PYZdq{}}\PY{l+s+s2}{\PYZdl{}y\PYZus{}}\PY{l+s+s2}{\PYZob{}}\PY{l+s+s2}{\PYZbs{}}\PY{l+s+s2}{mathrm}\PY{l+s+si}{\PYZob{}ex\PYZcb{}}\PY{l+s+s2}{\PYZcb{}\PYZdl{}}\PY{l+s+s2}{\PYZdq{}}\PY{p}{,} \PY{l+s+s2}{\PYZdq{}}\PY{l+s+s2}{\PYZdl{}y\PYZus{}}\PY{l+s+s2}{\PYZob{}}\PY{l+s+s2}{\PYZbs{}}\PY{l+s+s2}{mathrm}\PY{l+s+si}{\PYZob{}SSPRK\PYZus{}3\PYZcb{}}\PY{l+s+s2}{\PYZcb{}\PYZdl{}}\PY{l+s+s2}{\PYZdq{}} \PY{p}{]}\PY{p}{)}
\end{Verbatim}
\end{tcolorbox}

            \begin{tcolorbox}[breakable, size=fbox, boxrule=.5pt, pad at break*=1mm, opacityfill=0]
\prompt{Out}{outcolor}{6}{\boxspacing}
\begin{Verbatim}[commandchars=\\\{\}]
<matplotlib.legend.Legend at 0x7fb720486640>
\end{Verbatim}
\end{tcolorbox}
        
    \begin{center}
    \adjustimage{max size={0.9\linewidth}{0.9\paperheight}}{output_10_1.png}
    \end{center}
    { \hspace*{\fill} \\}
    
    \textbf{b)} We will now numerically investigate the RK-methods. We can
do this since we know what the exact solution to the ODE above is. We
assume that the error \(e = |y(T) - y_N|\) when using step size \(\tau\)
is approximately \[
e \approx C\tau^p
\] for some \(C>0\) and \(p.\) Note that \(p\) is what we call the
convergence order. We assume that \(p\) and \(C\) is the same when using
different step sizes \(h\). Let \(e_1\) and \(e_2\) be the errors when
using step sizes \(h_1\) and \(h_2\). Then we have

\[
\frac{e_1}{e_2} \approx \frac{\tau_1^p}{\tau_2^p} = \left(\frac{\tau_1}{\tau_2}\right)^p.
\] Taking logarithms on both sides we get \[
\log(e_1/e_2) \approx p \log(\tau_1/\tau_2)
\] or \$ \begin{align}
p \approx& \frac{\log(e_1/e_2)}{\log(\tau_1/\tau_2)}.
\end{align} \$ The value on the right-hand side of this equation is what
we call the Experimental Order of Convergence, or EOC. We will now try
to estimate the order of convergence using EOC-values.

Do the following for each method: 1. For m=0,\ldots,5, set
\(\tau_m=2^{-m}\) and find the value of \(N_{max}\) for each \(m\). 2.
Find the numerical solution \(y_{N(m)}\) of the ODE at \(T = 10\). 2.
Calculate the error \(e_m = |y(10) - y_{N_{max,m}}|\). 3. Calculate the
EOC for neighbouring step sizes, that is using equation (1) above with
\(e_m\) and \(e_{m+1}\) for \(m=0,...,4\). This should give you \(5\)
different approximations.

Draw a conclusion about the order of convergence \(p\) for each method.
Does it agree with the result in exercise 1?

    \begin{tcolorbox}[breakable, size=fbox, boxrule=1pt, pad at break*=1mm,colback=cellbackground, colframe=cellborder]
\prompt{In}{incolor}{7}{\boxspacing}
\begin{Verbatim}[commandchars=\\\{\}]
\PY{c+c1}{\PYZsh{} N\PYZus{}max will be the same for both methods.}
\PY{n}{m} \PY{o}{=} \PY{p}{[}\PY{n}{i} \PY{k}{for} \PY{n}{i} \PY{o+ow}{in} \PY{n+nb}{range}\PY{p}{(}\PY{l+m+mi}{6}\PY{p}{)}\PY{p}{]}
\PY{n}{t0} \PY{o}{=} \PY{l+m+mi}{0}
\PY{n}{T} \PY{o}{=} \PY{l+m+mi}{10}
\PY{n}{N\PYZus{}maxes} \PY{o}{=} \PY{p}{[}\PY{p}{]}
\PY{c+c1}{\PYZsh{} N\PYZus{}max = (T \PYZhy{} t0)*2\PYZca{}m}
\PY{k}{for} \PY{n}{i} \PY{o+ow}{in} \PY{n}{m}\PY{p}{:}
    \PY{n}{N\PYZus{}maxes}\PY{o}{.}\PY{n}{append}\PY{p}{(}\PY{p}{(}\PY{n}{T} \PY{o}{\PYZhy{}} \PY{n}{t0}\PY{p}{)} \PY{o}{*} \PY{l+m+mi}{2}\PY{o}{*}\PY{o}{*}\PY{n}{i}\PY{p}{)}

\PY{c+c1}{\PYZsh{} solving the ODE with different N\PYZus{}max}
\PY{c+c1}{\PYZsh{} dont need to store the result, only need to store the last value, eg y\PYZus{}N}
\PY{n}{ys\PYZus{}Nmax\PYZus{}mpr} \PY{o}{=} \PY{p}{[}\PY{p}{]}
\PY{n}{ys\PYZus{}Nmax\PYZus{}ssprk} \PY{o}{=} \PY{p}{[}\PY{p}{]}

\PY{k}{for} \PY{n}{N\PYZus{}m} \PY{o+ow}{in} \PY{n}{N\PYZus{}maxes}\PY{p}{:}
    \PY{n}{ts}\PY{p}{,} \PY{n}{ys} \PY{o}{=} \PY{n}{explicit\PYZus{}midpoint\PYZus{}rule}\PY{p}{(}\PY{n}{y0}\PY{p}{,} \PY{n}{t0}\PY{p}{,} \PY{n}{T}\PY{p}{,} \PY{n}{f}\PY{p}{,} \PY{n}{N\PYZus{}m}\PY{p}{)} \PY{c+c1}{\PYZsh{} f is the same as defined above}
    \PY{n}{ys\PYZus{}Nmax\PYZus{}mpr}\PY{o}{.}\PY{n}{append}\PY{p}{(}\PY{n}{ys}\PY{p}{[}\PY{o}{\PYZhy{}}\PY{l+m+mi}{1}\PY{p}{]}\PY{p}{)}
    \PY{n}{ts}\PY{p}{,} \PY{n}{ys} \PY{o}{=} \PY{n}{explicit\PYZus{}ssprk3}\PY{p}{(}\PY{n}{y0}\PY{p}{,} \PY{n}{t0}\PY{p}{,} \PY{n}{T}\PY{p}{,} \PY{n}{f}\PY{p}{,} \PY{n}{N\PYZus{}m}\PY{p}{)}
    \PY{n}{ys\PYZus{}Nmax\PYZus{}ssprk}\PY{o}{.}\PY{n}{append}\PY{p}{(}\PY{n}{ys}\PY{p}{[}\PY{o}{\PYZhy{}}\PY{l+m+mi}{1}\PY{p}{]}\PY{p}{)}

\PY{c+c1}{\PYZsh{} calculating the errors}
\PY{n}{e\PYZus{}m\PYZus{}mpr\PYZus{}list} \PY{o}{=} \PY{p}{[}\PY{p}{]}
\PY{n}{e\PYZus{}m\PYZus{}ssprk\PYZus{}list} \PY{o}{=} \PY{p}{[}\PY{p}{]}

\PY{k}{for} \PY{n}{i} \PY{o+ow}{in} \PY{n+nb}{range}\PY{p}{(}\PY{l+m+mi}{6}\PY{p}{)}\PY{p}{:}
    \PY{c+c1}{\PYZsh{} midpoint rule}
    \PY{n}{e\PYZus{}m\PYZus{}mpr} \PY{o}{=} \PY{n}{np}\PY{o}{.}\PY{n}{absolute}\PY{p}{(}\PY{n}{y\PYZus{}ex}\PY{p}{(}\PY{l+m+mi}{10}\PY{p}{)} \PY{o}{\PYZhy{}} \PY{n}{ys\PYZus{}Nmax\PYZus{}mpr}\PY{p}{[}\PY{n}{i}\PY{p}{]}\PY{p}{)}
    \PY{n}{e\PYZus{}m\PYZus{}mpr\PYZus{}list}\PY{o}{.}\PY{n}{append}\PY{p}{(}\PY{n}{e\PYZus{}m\PYZus{}mpr}\PY{p}{)}
    \PY{c+c1}{\PYZsh{} SSPRK3}
    \PY{n}{e\PYZus{}m\PYZus{}ssprk} \PY{o}{=} \PY{n}{np}\PY{o}{.}\PY{n}{absolute}\PY{p}{(}\PY{n}{y\PYZus{}ex}\PY{p}{(}\PY{l+m+mi}{10}\PY{p}{)} \PY{o}{\PYZhy{}} \PY{n}{ys\PYZus{}Nmax\PYZus{}ssprk}\PY{p}{[}\PY{n}{i}\PY{p}{]}\PY{p}{)}
    \PY{n}{e\PYZus{}m\PYZus{}ssprk\PYZus{}list}\PY{o}{.}\PY{n}{append}\PY{p}{(}\PY{n}{e\PYZus{}m\PYZus{}ssprk}\PY{p}{)}

\PY{c+c1}{\PYZsh{} finally calculating the EOC}
\PY{n}{eoc\PYZus{}mpr\PYZus{}list} \PY{o}{=} \PY{p}{[}\PY{p}{]}
\PY{n}{eoc\PYZus{}ssprk\PYZus{}list} \PY{o}{=} \PY{p}{[}\PY{p}{]}

\PY{k}{for} \PY{n}{i} \PY{o+ow}{in} \PY{n+nb}{range}\PY{p}{(}\PY{l+m+mi}{5}\PY{p}{)}\PY{p}{:}
    \PY{c+c1}{\PYZsh{} midpoint rule}
    \PY{n}{p\PYZus{}mpr\PYZus{}i} \PY{o}{=} \PY{p}{(}\PY{n}{np}\PY{o}{.}\PY{n}{log}\PY{p}{(}\PY{n}{e\PYZus{}m\PYZus{}mpr\PYZus{}list}\PY{p}{[}\PY{n}{i}\PY{p}{]}\PY{o}{/}\PY{n}{e\PYZus{}m\PYZus{}mpr\PYZus{}list}\PY{p}{[}\PY{n}{i}\PY{o}{+}\PY{l+m+mi}{1}\PY{p}{]}\PY{p}{)}\PY{p}{)}\PY{o}{/}\PY{p}{(}\PY{n}{np}\PY{o}{.}\PY{n}{log}\PY{p}{(}\PY{l+m+mi}{2}\PY{o}{*}\PY{o}{*}\PY{p}{(}\PY{o}{\PYZhy{}}\PY{n}{i}\PY{p}{)}\PY{o}{/}\PY{l+m+mi}{2}\PY{o}{*}\PY{o}{*}\PY{p}{(}\PY{o}{\PYZhy{}}\PY{n}{i}\PY{o}{\PYZhy{}}\PY{l+m+mi}{1}\PY{p}{)}\PY{p}{)}\PY{p}{)}
    \PY{n}{eoc\PYZus{}mpr\PYZus{}list}\PY{o}{.}\PY{n}{append}\PY{p}{(}\PY{n}{p\PYZus{}mpr\PYZus{}i}\PY{p}{)}
    \PY{c+c1}{\PYZsh{} SSPRK3}
    \PY{n}{p\PYZus{}ssprk\PYZus{}i} \PY{o}{=} \PY{p}{(}\PY{n}{np}\PY{o}{.}\PY{n}{log}\PY{p}{(}\PY{n}{e\PYZus{}m\PYZus{}ssprk\PYZus{}list}\PY{p}{[}\PY{n}{i}\PY{p}{]}\PY{o}{/}\PY{n}{e\PYZus{}m\PYZus{}ssprk\PYZus{}list}\PY{p}{[}\PY{n}{i}\PY{o}{+}\PY{l+m+mi}{1}\PY{p}{]}\PY{p}{)}\PY{p}{)}\PY{o}{/}\PY{p}{(}\PY{n}{np}\PY{o}{.}\PY{n}{log}\PY{p}{(}\PY{l+m+mi}{2}\PY{o}{*}\PY{o}{*}\PY{p}{(}\PY{o}{\PYZhy{}}\PY{n}{i}\PY{p}{)}\PY{o}{/}\PY{l+m+mi}{2}\PY{o}{*}\PY{o}{*}\PY{p}{(}\PY{o}{\PYZhy{}}\PY{n}{i}\PY{o}{\PYZhy{}}\PY{l+m+mi}{1}\PY{p}{)}\PY{p}{)}\PY{p}{)}
    \PY{n}{eoc\PYZus{}ssprk\PYZus{}list}\PY{o}{.}\PY{n}{append}\PY{p}{(}\PY{n}{p\PYZus{}ssprk\PYZus{}i}\PY{p}{)}

\PY{n+nb}{print}\PY{p}{(}\PY{l+s+s2}{\PYZdq{}}\PY{l+s+s2}{Midpoint rule:}\PY{l+s+s2}{\PYZdq{}}\PY{p}{)}
\PY{k}{for} \PY{n}{i} \PY{o+ow}{in} \PY{n+nb}{range}\PY{p}{(}\PY{l+m+mi}{5}\PY{p}{)}\PY{p}{:}
    \PY{n+nb}{print}\PY{p}{(}\PY{l+s+sa}{f}\PY{l+s+s2}{\PYZdq{}}\PY{l+s+s2}{tau\PYZus{}}\PY{l+s+si}{\PYZob{}}\PY{n}{i}\PY{l+s+si}{\PYZcb{}}\PY{l+s+s2}{ = }\PY{l+s+si}{\PYZob{}}\PY{l+m+mi}{2}\PY{o}{*}\PY{o}{*}\PY{p}{(}\PY{o}{\PYZhy{}}\PY{n}{i}\PY{p}{)}\PY{l+s+si}{\PYZcb{}}\PY{l+s+s2}{,tau\PYZus{}}\PY{l+s+si}{\PYZob{}}\PY{n}{i}\PY{o}{+}\PY{l+m+mi}{1}\PY{l+s+si}{\PYZcb{}}\PY{l+s+s2}{ = }\PY{l+s+si}{\PYZob{}}\PY{l+m+mi}{2}\PY{o}{*}\PY{o}{*}\PY{p}{(}\PY{o}{\PYZhy{}}\PY{n}{i}\PY{o}{\PYZhy{}}\PY{l+m+mi}{1}\PY{p}{)}\PY{l+s+si}{\PYZcb{}}\PY{l+s+s2}{ gives }\PY{l+s+s2}{\PYZdq{}}
          \PY{o}{+} \PY{l+s+sa}{f}\PY{l+s+s2}{\PYZdq{}}\PY{l+s+s2}{experimental order of convergence p = }\PY{l+s+si}{\PYZob{}}\PY{n}{eoc\PYZus{}mpr\PYZus{}list}\PY{p}{[}\PY{n}{i}\PY{p}{]}\PY{l+s+si}{\PYZcb{}}\PY{l+s+s2}{\PYZdq{}}\PY{p}{)}

\PY{n+nb}{print}\PY{p}{(}\PY{l+s+s2}{\PYZdq{}}\PY{l+s+se}{\PYZbs{}n}\PY{l+s+s2}{SSPRK3:}\PY{l+s+s2}{\PYZdq{}}\PY{p}{)}
\PY{k}{for} \PY{n}{i} \PY{o+ow}{in} \PY{n+nb}{range}\PY{p}{(}\PY{l+m+mi}{5}\PY{p}{)}\PY{p}{:}
    \PY{n+nb}{print}\PY{p}{(}\PY{l+s+sa}{f}\PY{l+s+s2}{\PYZdq{}}\PY{l+s+s2}{tau\PYZus{}}\PY{l+s+si}{\PYZob{}}\PY{n}{i}\PY{l+s+si}{\PYZcb{}}\PY{l+s+s2}{ = }\PY{l+s+si}{\PYZob{}}\PY{l+m+mi}{2}\PY{o}{*}\PY{o}{*}\PY{p}{(}\PY{o}{\PYZhy{}}\PY{n}{i}\PY{p}{)}\PY{l+s+si}{\PYZcb{}}\PY{l+s+s2}{,tau\PYZus{}}\PY{l+s+si}{\PYZob{}}\PY{n}{i}\PY{o}{+}\PY{l+m+mi}{1}\PY{l+s+si}{\PYZcb{}}\PY{l+s+s2}{ = }\PY{l+s+si}{\PYZob{}}\PY{l+m+mi}{2}\PY{o}{*}\PY{o}{*}\PY{p}{(}\PY{o}{\PYZhy{}}\PY{n}{i}\PY{o}{\PYZhy{}}\PY{l+m+mi}{1}\PY{p}{)}\PY{l+s+si}{\PYZcb{}}\PY{l+s+s2}{ gives }\PY{l+s+s2}{\PYZdq{}}
          \PY{o}{+} \PY{l+s+sa}{f}\PY{l+s+s2}{\PYZdq{}}\PY{l+s+s2}{experimental order of convergence p = }\PY{l+s+si}{\PYZob{}}\PY{n}{eoc\PYZus{}ssprk\PYZus{}list}\PY{p}{[}\PY{n}{i}\PY{p}{]}\PY{l+s+si}{\PYZcb{}}\PY{l+s+s2}{\PYZdq{}}\PY{p}{)}
\end{Verbatim}
\end{tcolorbox}

    \begin{Verbatim}[commandchars=\\\{\}]
Midpoint rule:
tau\_0 = 1,tau\_1 = 0.5 gives experimental order of convergence p =
4.641084409856239
tau\_1 = 0.5,tau\_2 = 0.25 gives experimental order of convergence p =
2.618767003133043
tau\_2 = 0.25,tau\_3 = 0.125 gives experimental order of convergence p =
2.2053567418841813
tau\_3 = 0.125,tau\_4 = 0.0625 gives experimental order of convergence p =
2.0834717759359433
tau\_4 = 0.0625,tau\_5 = 0.03125 gives experimental order of convergence p =
2.0375876961837682

SSPRK3:
tau\_0 = 1,tau\_1 = 0.5 gives experimental order of convergence p =
3.0609310003784023
tau\_1 = 0.5,tau\_2 = 0.25 gives experimental order of convergence p =
3.244464219011232
tau\_2 = 0.25,tau\_3 = 0.125 gives experimental order of convergence p =
3.1402030351845105
tau\_3 = 0.125,tau\_4 = 0.0625 gives experimental order of convergence p =
3.0717942133260796
tau\_4 = 0.0625,tau\_5 = 0.03125 gives experimental order of convergence p =
3.036056209583162
    \end{Verbatim}

    \textbf{c)} We will finally test both methods on the ODE

\begin{align*}
y'(t) =& -2ty(t), &\quad y(0) = 1, &\quad t \in [0,0.5].
\end{align*}

This has exact solution \(e^{-t^2}\). Find the approximate value of
\(y(0.5)\) using * The midpoint method with \(N_{max} = 3\) * The SSPRK3
method with \(N_{max} = 2\)

The number of step sizes are chosen such that each method needs to
perform 6 evaluations of the function \(f\). How do the errors
\(e = |y(T) - y_{N_{max}}|\) compare?

\emph{Additional exercise:} Does this observation holds for other values
of \(T\)? For instance, with \(T=0.2\) or \(T=0.8\). Can you tell what
happens to \(y\) or \(y^\prime\) in \(T=0.5\)?

    \begin{tcolorbox}[breakable, size=fbox, boxrule=1pt, pad at break*=1mm,colback=cellbackground, colframe=cellborder]
\prompt{In}{incolor}{8}{\boxspacing}
\begin{Verbatim}[commandchars=\\\{\}]
\PY{c+c1}{\PYZsh{} defining the exact solution}
\PY{k}{def} \PY{n+nf}{y\PYZus{}ex}\PY{p}{(}\PY{n}{t}\PY{p}{)}\PY{p}{:}
    \PY{k}{return} \PY{n}{np}\PY{o}{.}\PY{n}{exp}\PY{p}{(}\PY{o}{\PYZhy{}}\PY{n}{t}\PY{o}{*}\PY{o}{*}\PY{l+m+mi}{2}\PY{p}{)}

\PY{c+c1}{\PYZsh{} defining g (right hand side of the equation)}
\PY{k}{def} \PY{n+nf}{g}\PY{p}{(}\PY{n}{t}\PY{p}{,} \PY{n}{y}\PY{p}{)}\PY{p}{:}
    \PY{k}{return} \PY{o}{\PYZhy{}}\PY{l+m+mi}{2}\PY{o}{*}\PY{n}{t}\PY{o}{*}\PY{n}{y}

\PY{c+c1}{\PYZsh{} initial values}
\PY{n}{y0} \PY{o}{=} \PY{l+m+mi}{1}
\PY{n}{t0} \PY{o}{=} \PY{l+m+mi}{0}
\PY{n}{T} \PY{o}{=} \PY{l+m+mf}{0.5}
\PY{n}{N\PYZus{}max\PYZus{}mpr} \PY{o}{=} \PY{l+m+mi}{3}
\PY{n}{N\PYZus{}max\PYZus{}ssprk} \PY{o}{=} \PY{l+m+mi}{2}

\PY{c+c1}{\PYZsh{} solving the IVP}
\PY{n}{ts\PYZus{}mpr}\PY{p}{,} \PY{n}{ys\PYZus{}mpr} \PY{o}{=} \PY{n}{explicit\PYZus{}midpoint\PYZus{}rule}\PY{p}{(}\PY{n}{y0}\PY{p}{,} \PY{n}{t0}\PY{p}{,} \PY{n}{T}\PY{p}{,} \PY{n}{g}\PY{p}{,} \PY{n}{N\PYZus{}max\PYZus{}mpr}\PY{p}{)}
\PY{n}{ts\PYZus{}ssprk}\PY{p}{,} \PY{n}{ys\PYZus{}ssprk} \PY{o}{=} \PY{n}{explicit\PYZus{}ssprk3}\PY{p}{(}\PY{n}{y0}\PY{p}{,} \PY{n}{t0}\PY{p}{,} \PY{n}{T}\PY{p}{,} \PY{n}{g}\PY{p}{,} \PY{n}{N\PYZus{}max\PYZus{}ssprk}\PY{p}{)}

\PY{n}{x} \PY{o}{=} \PY{n}{np}\PY{o}{.}\PY{n}{linspace}\PY{p}{(}\PY{l+m+mi}{0}\PY{p}{,} \PY{l+m+mf}{0.5}\PY{p}{,} \PY{l+m+mi}{5}\PY{p}{)}
\PY{n}{ys\PYZus{}ex} \PY{o}{=} \PY{p}{[}\PY{p}{]}
\PY{k}{for} \PY{n}{val} \PY{o+ow}{in} \PY{n}{x}\PY{p}{:}
    \PY{n}{ys\PYZus{}ex}\PY{o}{.}\PY{n}{append}\PY{p}{(}\PY{n}{y\PYZus{}ex}\PY{p}{(}\PY{n}{val}\PY{p}{)}\PY{p}{)}


\PY{n}{fig}\PY{p}{,} \PY{n}{axes} \PY{o}{=} \PY{n}{plt}\PY{o}{.}\PY{n}{subplots}\PY{p}{(}\PY{l+m+mi}{1}\PY{p}{,}\PY{l+m+mi}{2}\PY{p}{)}
\PY{n}{axes}\PY{p}{[}\PY{l+m+mi}{0}\PY{p}{]}\PY{o}{.}\PY{n}{plot}\PY{p}{(}\PY{n}{x}\PY{p}{,} \PY{n}{ys\PYZus{}ex}\PY{p}{,} \PY{l+s+s1}{\PYZsq{}}\PY{l+s+s1}{bo\PYZhy{}}\PY{l+s+s1}{\PYZsq{}}\PY{p}{)}
\PY{n}{axes}\PY{p}{[}\PY{l+m+mi}{0}\PY{p}{]}\PY{o}{.}\PY{n}{plot}\PY{p}{(}\PY{n}{ts\PYZus{}mpr}\PY{p}{,} \PY{n}{ys\PYZus{}mpr}\PY{p}{,} \PY{l+s+s1}{\PYZsq{}}\PY{l+s+s1}{rx\PYZhy{}}\PY{l+s+s1}{\PYZsq{}}\PY{p}{)}
\PY{n}{axes}\PY{p}{[}\PY{l+m+mi}{0}\PY{p}{]}\PY{o}{.}\PY{n}{legend}\PY{p}{(}\PY{p}{[}\PY{l+s+s2}{\PYZdq{}}\PY{l+s+s2}{\PYZdl{}y\PYZus{}}\PY{l+s+s2}{\PYZob{}}\PY{l+s+s2}{\PYZbs{}}\PY{l+s+s2}{mathrm}\PY{l+s+si}{\PYZob{}ex\PYZcb{}}\PY{l+s+s2}{\PYZcb{}\PYZdl{}}\PY{l+s+s2}{\PYZdq{}}\PY{p}{,} \PY{l+s+s2}{\PYZdq{}}\PY{l+s+s2}{\PYZdl{}y\PYZus{}}\PY{l+s+s2}{\PYZob{}}\PY{l+s+s2}{\PYZbs{}}\PY{l+s+s2}{mathrm}\PY{l+s+si}{\PYZob{}mpr\PYZcb{}}\PY{l+s+s2}{\PYZcb{}\PYZdl{}}\PY{l+s+s2}{\PYZdq{}} \PY{p}{]}\PY{p}{)}
\PY{n}{ts}\PY{p}{,} \PY{n}{ys\PYZus{}ssprk3} \PY{o}{=} \PY{n}{explicit\PYZus{}ssprk3}\PY{p}{(}\PY{n}{y0}\PY{p}{,} \PY{n}{t0}\PY{p}{,} \PY{n}{T}\PY{p}{,} \PY{n}{f}\PY{p}{,} \PY{n}{N\PYZus{}max}\PY{p}{)}
\PY{n}{axes}\PY{p}{[}\PY{l+m+mi}{1}\PY{p}{]}\PY{o}{.}\PY{n}{plot}\PY{p}{(}\PY{n}{x}\PY{p}{,} \PY{n}{ys\PYZus{}ex}\PY{p}{,} \PY{l+s+s1}{\PYZsq{}}\PY{l+s+s1}{bo\PYZhy{}}\PY{l+s+s1}{\PYZsq{}}\PY{p}{)}
\PY{n}{axes}\PY{p}{[}\PY{l+m+mi}{1}\PY{p}{]}\PY{o}{.}\PY{n}{plot}\PY{p}{(}\PY{n}{ts\PYZus{}ssprk}\PY{p}{,} \PY{n}{ys\PYZus{}ssprk}\PY{p}{,} \PY{l+s+s1}{\PYZsq{}}\PY{l+s+s1}{rx\PYZhy{}}\PY{l+s+s1}{\PYZsq{}}\PY{p}{)}
\PY{n}{axes}\PY{p}{[}\PY{l+m+mi}{1}\PY{p}{]}\PY{o}{.}\PY{n}{legend}\PY{p}{(}\PY{p}{[}\PY{l+s+s2}{\PYZdq{}}\PY{l+s+s2}{\PYZdl{}y\PYZus{}}\PY{l+s+s2}{\PYZob{}}\PY{l+s+s2}{\PYZbs{}}\PY{l+s+s2}{mathrm}\PY{l+s+si}{\PYZob{}ex\PYZcb{}}\PY{l+s+s2}{\PYZcb{}\PYZdl{}}\PY{l+s+s2}{\PYZdq{}}\PY{p}{,} \PY{l+s+s2}{\PYZdq{}}\PY{l+s+s2}{\PYZdl{}y\PYZus{}}\PY{l+s+s2}{\PYZob{}}\PY{l+s+s2}{\PYZbs{}}\PY{l+s+s2}{mathrm}\PY{l+s+si}{\PYZob{}SSPRK\PYZus{}3\PYZcb{}}\PY{l+s+s2}{\PYZcb{}\PYZdl{}}\PY{l+s+s2}{\PYZdq{}} \PY{p}{]}\PY{p}{)}

\PY{n}{e\PYZus{}mpr} \PY{o}{=} \PY{n}{np}\PY{o}{.}\PY{n}{absolute}\PY{p}{(}\PY{n}{y\PYZus{}ex}\PY{p}{(}\PY{n}{T}\PY{p}{)} \PY{o}{\PYZhy{}} \PY{n}{ys\PYZus{}mpr}\PY{p}{[}\PY{o}{\PYZhy{}}\PY{l+m+mi}{1}\PY{p}{]}\PY{p}{)}
\PY{n}{e\PYZus{}ssprk} \PY{o}{=} \PY{n}{np}\PY{o}{.}\PY{n}{absolute}\PY{p}{(}\PY{n}{y\PYZus{}ex}\PY{p}{(}\PY{n}{T}\PY{p}{)} \PY{o}{\PYZhy{}} \PY{n}{ys\PYZus{}ssprk}\PY{p}{[}\PY{o}{\PYZhy{}}\PY{l+m+mi}{1}\PY{p}{]}\PY{p}{)}

\PY{n+nb}{print}\PY{p}{(}\PY{l+s+sa}{f}\PY{l+s+s2}{\PYZdq{}}\PY{l+s+s2}{Error for midpoint rule: e = }\PY{l+s+si}{\PYZob{}}\PY{n}{e\PYZus{}mpr}\PY{l+s+si}{\PYZcb{}}\PY{l+s+s2}{\PYZdq{}}\PY{p}{)}
\PY{n+nb}{print}\PY{p}{(}\PY{l+s+sa}{f}\PY{l+s+s2}{\PYZdq{}}\PY{l+s+s2}{Error for SSPRK3: e = }\PY{l+s+si}{\PYZob{}}\PY{n}{e\PYZus{}ssprk}\PY{l+s+si}{\PYZcb{}}\PY{l+s+s2}{\PYZdq{}}\PY{p}{)}
\end{Verbatim}
\end{tcolorbox}

    \begin{Verbatim}[commandchars=\\\{\}]
Error for midpoint rule: e = 0.002543491722271085
Error for SSPRK3: e = 0.0010497081744864634
    \end{Verbatim}

    \begin{center}
    \adjustimage{max size={0.9\linewidth}{0.9\paperheight}}{output_14_1.png}
    \end{center}
    { \hspace*{\fill} \\}
    
    \hypertarget{exercise-3}{%
\subsubsection{\texorpdfstring{\textbf{Exercise
3}}{Exercise 3}}\label{exercise-3}}

\hypertarget{sir-model}{%
\paragraph{SIR Model}\label{sir-model}}

The SIR model is a system of first order ODE's which model the dynamics
of a disease in a society.

There are \(S(t)\) susceptible/healthy individuals, \(I(t)\) infected
individuals and \(R(t)\) recovered individuals. Each susceptible person
has a risk of becoming infected, a risk which is proportional to the
number of infected people \(I(t)\), with a proportinality constant
\(\beta>0.\) Each infected person also has a chance of recovering, with
a recovery constant \(\gamma>0\). This leads to the coupled system of
first-order ODE's

\[
\begin{align*}
S'(t) = &  -\beta S(t) I(t),\\
I'(t) = &  \beta S(t) I(t) - \gamma I(t),\\
R'(t) = &  \gamma I(t).
\end{align*}
\]

We can rewrite this in vector form as

\[
\mathbf{u}'(t) = \mathbf{f}(\mathbf{u}(t))
\]

where we have defined

\[
\mathbf{u}(t) = 
\begin{pmatrix}
S(t)\\
I(t)\\
R(t)
\end{pmatrix}, \quad 
\mathbf{f}(\mathbf{u}(t)) = 
\begin{pmatrix}
-\beta S(t) I(t)\\
\beta S(t) I(t) - \gamma I(t)\\
\gamma I(t)
\end{pmatrix}.
\]

    \textbf{a)} Show that the system is conservative, that is that the total
number of individuals \(S(t)+I(t)+R(t)\) is constant.

\emph{Hint:} remember which is the derivative of a constant function.

    If we add the 3 equations we will get:
\[S'(t) + I'(t) + R'(t) = - \beta S(t)I(t) + \beta S(t)I(t) - \gamma I(t) + \gamma I(t)\]
\[S'(t) + I'(t) + R'(t) = 0 \] If we now integrate on both sides we will
get: \[ S(t) + I(t) + R(t) = \int_0^{\infty} 0 dt\]
\[ S(t) + I(t) + R(t) = C \], where C is a constant, \(S\) is the
antiderivative of \(S'(t)\), \(I\) is the antiderivative of \(I'(t)\)
and \(R\) is the antiderivative of \(R'(t)\). This shows that the sum of
these 3 functions will always be a constant.

    \textbf{b)} Numerically solve the system \[
\mathbf{u}'(t) = \mathbf{f}(\mathbf{u}(t)),\quad \mathbf{u}(0) = \mathbf{u}_0,\quad t\in [0,T]
\] with either of the RK-methods above.

\begin{itemize}
\tightlist
\item
  You may pick end-time \(T\) and \(N_{max}\),
\item
  Choose an initial number of individuals.

  \begin{itemize}
  \tightlist
  \item
    A suitable inital conditions could e.g.~be
    \(\mathbf{u}_0 = (50,10,0)^T\).
  \end{itemize}
\item
  Plot the solution as a function of time.
\item
  Also plot the total number of individuals.

  \begin{itemize}
  \tightlist
  \item
    Is the total number conserved? To check this, you might calculate
    the maximum total and the minimum total over the interval.
  \end{itemize}
\item
  The parameters \(\beta=0.2\) and \(\gamma=0.15\) can be helpful for
  ilustration.
\end{itemize}

If you need a guide for this problem, you can look at the Lotka-Volterra
model. It was presented in the lectures and is also available in the
learning material on the wiki-page.

    \begin{tcolorbox}[breakable, size=fbox, boxrule=1pt, pad at break*=1mm,colback=cellbackground, colframe=cellborder]
\prompt{In}{incolor}{9}{\boxspacing}
\begin{Verbatim}[commandchars=\\\{\}]
\PY{n}{T} \PY{o}{=} \PY{l+m+mi}{50}
\PY{n}{t0} \PY{o}{=} \PY{l+m+mi}{0}
\PY{n}{I0} \PY{o}{=} \PY{l+m+mi}{1}\PY{o}{/}\PY{l+m+mi}{200000}
\PY{n}{S0} \PY{o}{=} \PY{l+m+mi}{1} \PY{o}{\PYZhy{}} \PY{n}{I0}
\PY{n}{R0} \PY{o}{=} \PY{l+m+mi}{0}
\PY{n}{u0} \PY{o}{=} \PY{n}{np}\PY{o}{.}\PY{n}{array}\PY{p}{(}\PY{p}{[}\PY{l+m+mi}{50}\PY{p}{,} \PY{l+m+mi}{10}\PY{p}{,} \PY{l+m+mi}{0}\PY{p}{]}\PY{p}{)}
\PY{n}{N\PYZus{}max} \PY{o}{=} \PY{l+m+mi}{150}
\PY{n}{beta} \PY{o}{=} \PY{l+m+mf}{0.1}
\PY{n}{gamma} \PY{o}{=} \PY{l+m+mf}{0.05}

\PY{k}{def} \PY{n+nf}{f}\PY{p}{(}\PY{n}{t}\PY{p}{,} \PY{n}{u}\PY{p}{)}\PY{p}{:}
    \PY{k}{return} \PY{n}{np}\PY{o}{.}\PY{n}{array}\PY{p}{(}\PY{p}{[}\PY{o}{\PYZhy{}}\PY{n}{beta}\PY{o}{*}\PY{n}{u}\PY{p}{[}\PY{l+m+mi}{0}\PY{p}{]}\PY{o}{*}\PY{n}{u}\PY{p}{[}\PY{l+m+mi}{1}\PY{p}{]}\PY{p}{,}
                     \PY{n}{beta}\PY{o}{*}\PY{n}{u}\PY{p}{[}\PY{l+m+mi}{0}\PY{p}{]}\PY{o}{*}\PY{n}{u}\PY{p}{[}\PY{l+m+mi}{1}\PY{p}{]} \PY{o}{\PYZhy{}} \PY{n}{gamma}\PY{o}{*}\PY{n}{u}\PY{p}{[}\PY{l+m+mi}{1}\PY{p}{]}\PY{p}{,}
                     \PY{n}{gamma} \PY{o}{*} \PY{n}{u}\PY{p}{[}\PY{l+m+mi}{1}\PY{p}{]}
                    \PY{p}{]}\PY{p}{)}

\PY{k}{def} \PY{n+nf}{totalPopulation}\PY{p}{(}\PY{n}{u}\PY{p}{)}\PY{p}{:}
    \PY{k}{return} \PY{n}{u}\PY{p}{[}\PY{l+m+mi}{0}\PY{p}{]} \PY{o}{+} \PY{n}{u}\PY{p}{[}\PY{l+m+mi}{1}\PY{p}{]} \PY{o}{+} \PY{n}{u}\PY{p}{[}\PY{l+m+mi}{2}\PY{p}{]}

\PY{n}{ts}\PY{p}{,} \PY{n}{ys} \PY{o}{=} \PY{n}{explicit\PYZus{}ssprk3}\PY{p}{(}\PY{n}{u0}\PY{p}{,} \PY{n}{t0}\PY{p}{,} \PY{n}{T}\PY{p}{,} \PY{n}{f}\PY{p}{,} \PY{n}{N\PYZus{}max}\PY{p}{)}
\PY{n}{pop} \PY{o}{=} \PY{n}{np}\PY{o}{.}\PY{n}{zeros}\PY{p}{(}\PY{n+nb}{len}\PY{p}{(}\PY{n}{ys}\PY{p}{)}\PY{p}{)}
\PY{n}{ts\PYZus{}pop} \PY{o}{=} \PY{n}{np}\PY{o}{.}\PY{n}{zeros}\PY{p}{(}\PY{n+nb}{len}\PY{p}{(}\PY{n}{ys}\PY{p}{)}\PY{p}{)}
\PY{k}{for} \PY{n}{i} \PY{o+ow}{in} \PY{n+nb}{range}\PY{p}{(}\PY{n+nb}{len}\PY{p}{(}\PY{n}{ys}\PY{p}{)}\PY{p}{)}\PY{p}{:}
    \PY{n}{ts\PYZus{}pop}\PY{p}{[}\PY{n}{i}\PY{p}{]} \PY{o}{=} \PY{n}{ts}\PY{p}{[}\PY{n}{i}\PY{p}{]}
    \PY{n}{pop}\PY{p}{[}\PY{n}{i}\PY{p}{]} \PY{o}{=} \PY{n}{totalPopulation}\PY{p}{(}\PY{n}{ys}\PY{p}{[}\PY{n}{i}\PY{p}{]}\PY{p}{)}

\PY{n}{newparams}\PY{p}{[}\PY{l+s+s1}{\PYZsq{}}\PY{l+s+s1}{figure.figsize}\PY{l+s+s1}{\PYZsq{}}\PY{p}{]} \PY{o}{=} \PY{p}{(}\PY{l+m+mi}{8}\PY{p}{,}\PY{l+m+mi}{6}\PY{p}{)}
\PY{n}{plt}\PY{o}{.}\PY{n}{rcParams}\PY{o}{.}\PY{n}{update}\PY{p}{(}\PY{n}{newparams}\PY{p}{)}
\PY{n}{plt}\PY{o}{.}\PY{n}{plot}\PY{p}{(}\PY{n}{ts}\PY{p}{,} \PY{n}{ys}\PY{p}{,} \PY{l+s+s2}{\PYZdq{}}\PY{l+s+s2}{\PYZhy{}\PYZhy{}}\PY{l+s+s2}{\PYZdq{}}\PY{p}{)}
\PY{n}{plt}\PY{o}{.}\PY{n}{plot}\PY{p}{(}\PY{n}{ts\PYZus{}pop}\PY{p}{,} \PY{n}{pop}\PY{p}{,} \PY{l+s+s2}{\PYZdq{}}\PY{l+s+s2}{\PYZhy{}}\PY{l+s+s2}{\PYZdq{}}\PY{p}{)}
\PY{n}{plt}\PY{o}{.}\PY{n}{legend}\PY{p}{(}\PY{p}{[}\PY{l+s+s2}{\PYZdq{}}\PY{l+s+s2}{S}\PY{l+s+s2}{\PYZdq{}}\PY{p}{,} \PY{l+s+s2}{\PYZdq{}}\PY{l+s+s2}{I}\PY{l+s+s2}{\PYZdq{}}\PY{p}{,} \PY{l+s+s2}{\PYZdq{}}\PY{l+s+s2}{R}\PY{l+s+s2}{\PYZdq{}}\PY{p}{,} \PY{l+s+s2}{\PYZdq{}}\PY{l+s+s2}{Pop}\PY{l+s+s2}{\PYZdq{}}\PY{p}{]}\PY{p}{)}
\PY{n}{plt}\PY{o}{.}\PY{n}{xlabel}\PY{p}{(}\PY{l+s+s2}{\PYZdq{}}\PY{l+s+s2}{Days}\PY{l+s+s2}{\PYZdq{}}\PY{p}{)}
\PY{n}{plt}\PY{o}{.}\PY{n}{ylabel}\PY{p}{(}\PY{l+s+s2}{\PYZdq{}}\PY{l+s+s2}{Number of people}\PY{l+s+s2}{\PYZdq{}}\PY{p}{)}
\end{Verbatim}
\end{tcolorbox}

            \begin{tcolorbox}[breakable, size=fbox, boxrule=.5pt, pad at break*=1mm, opacityfill=0]
\prompt{Out}{outcolor}{9}{\boxspacing}
\begin{Verbatim}[commandchars=\\\{\}]
Text(0, 0.5, 'Number of people')
\end{Verbatim}
\end{tcolorbox}
        
    \begin{center}
    \adjustimage{max size={0.9\linewidth}{0.9\paperheight}}{output_19_1.png}
    \end{center}
    { \hspace*{\fill} \\}
    
    \emph{Additional exercise:} modify the SIR model such that the
population is not longer constant. One idea is to have a proportion of
the infected population to die with rate \(\delta\). Test the new model
as before.

    \begin{tcolorbox}[breakable, size=fbox, boxrule=1pt, pad at break*=1mm,colback=cellbackground, colframe=cellborder]
\prompt{In}{incolor}{ }{\boxspacing}
\begin{Verbatim}[commandchars=\\\{\}]

\end{Verbatim}
\end{tcolorbox}


    % Add a bibliography block to the postdoc
    
    
    
\end{document}
